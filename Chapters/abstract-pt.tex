%!TEX root = ../template.tex
%%%%%%%%%%%%%%%%%%%%%%%%%%%%%%%%%%%%%%%%%%%%%%%%%%%%%%%%%%%%%%%%%%%%
%% abstract-pt.tex
%% NOVA thesis document file
%%
%% Abstract in Portuguese
%%%%%%%%%%%%%%%%%%%%%%%%%%%%%%%%%%%%%%%%%%%%%%%%%%%%%%%%%%%%%%%%%%%%

\typeout{NT FILE abstract-pt.tex}%

% Esta tese apresenta um conjunto de técnicas de controlo com baixos requerimentos em termos computacionais, sustentadas pelas muito estudadas \glsxtrshort{CLF-CBF}, que através da otimização de um único parâmetro da condição de segurança visada pela \glsxtrshort{CBF}, são capazes de se adaptar melhor aos diferentes cenários encontrados. Entre as técnicas de controlo apresentadas, também é mostrado como a alteração do objetivo de estabilidade para um ponto transitivo calculado no limite do conjunto composto por estados de segurança precavem em certa parte contra uma potencial convergência para pontos de equilíbrio indesejados e melhora significativamente o caminho percorrido, uma vez que mitiga as aproximações desnecessárias por parte da \glsxtrshort{CLF-CBF} ao conjunto de estados restringidos promotores de inputs maiores, sem diminuir a sua velocidade de convergência para o ponto de equilíbrio desejado. Os controladores são testados em diferentes cenários de colisão e para diferentes sistemas, incluindo um monociclo e uma nave espacial. 

Esta tese tem o foco no desenvolvimento de estratégias de controlo de baixo custo computacional baseadas em \glsxtrfull{CLF-CBF} para cenários na eminencia de colisão. A \glsxtrshort{CLF-CBF} demonstrou ser um método de computação rápido que proporciona segurança e direciona o sistema para um local desejado. No entanto, observou-se que este método de controlo aproxima relativamente perto o sistema dos obstáculos, o que pode resultar em convergências para pontos indesejados nas suas proximidades. Além disso, esta abordagem revelou uma falta de adaptabilidade, evidenciado pelas trajetórias inadequadas que são adotadas para evitar obstáculos variados.
A presente dissertação apresenta uma série de algoritmos tendo a \glsxtrshort{CLF-CBF} (de baixo esforço) como método central. Estes algoritmos são capazes de alcançar um adequação superior a cenários variáveis encontrados, otimizando um único parâmetro da condição \glsxtrshort{CBF}. No contexto das técnicas de controlo apresentadas, é demonstrado que a modificação do objetivo de estabilidade para um ponto transitivo calculado no limite do obstáculo pode impedir a convergência para pontos de equilíbrio indesejáveis e melhorar significativamente a trajetória. Isto é, mitigando as aproximações desnecessárias do \glsxtrshort{CLF-CBF} ao obstaculo, que resultam em ações de controlo de maior magnitude, sem comprometer a taxa de convergência para o ponto de equilíbrio desejado. 
Os controladores são submetidos a testes rigorosos em uma série de cenários de colisão e em diversos sistemas, incluindo monociclos e naves espaciais.





% \keywords{
%   Primeira palavra-chave \and
%   Outra palavra-chave \and
%   Mais uma palavra-chave \and
%   A última palavra-chave
% }
