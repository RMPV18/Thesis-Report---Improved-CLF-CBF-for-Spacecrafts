%!TEX root = ../template.tex
%%%%%%%%%%%%%%%%%%%%%%%%%%%%%%%%%%%%%%%%%%%%%%%%%%%%%%%%%%%%%%%%%%%%
%% abstract-pt.tex
%% NOVA thesis document file
%%
%% Abstract in Portuguese
%%%%%%%%%%%%%%%%%%%%%%%%%%%%%%%%%%%%%%%%%%%%%%%%%%%%%%%%%%%%%%%%%%%%

\typeout{NT FILE abstract-pt.tex}%

Esta tese apresenta um conjunto de técnicas de controlo com baixos requerimentos em termos computacionais, sustentadas pelas muito estudadas \glsxtrshort{CLF-CBF}, que através da otimização de um único parâmetro da condição de segurança visada pela \glsxtrshort{CBF}, são capazes de se adaptar melhor aos diferentes cenários encontrados. Entre as técnicas de controlo apresentadas, também é mostrado como a alteração do objetivo de estabilidade para um ponto transitivo calculado no limite do conjunto composto por estados se segurança precavem em certa parte contra potencial convergência para pontos de equilíbrio indesejados e melhora significativamente o caminho percorrido, uma vez que mitiga as aproximações desnecessárias por parte da \glsxtrshort{CLF-CBF} ao conjunto de estados restringidos promotores de inputs maiores, sem diminuir a sua velocidade de convergência para o ponto de equilíbrio desejado. Os controladores são testados em diferentes cenários de colisão e para diferentes sistemas, incluindo um monociclo e uma nave espacial. 



% \keywords{
%   Primeira palavra-chave \and
%   Outra palavra-chave \and
%   Mais uma palavra-chave \and
%   A última palavra-chave
% }
