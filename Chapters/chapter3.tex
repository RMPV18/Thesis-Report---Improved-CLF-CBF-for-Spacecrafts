%!TEX root = ../template.tex
%%%%%%%%%%%%%%%%%%%%%%%%%%%%%%%%%%%%%%%%%%%%%%%%%%%%%%%%%%%%%%%%%%%%
%% chapter3.tex
%% NOVA thesis document file
%%
%% Chapter with a short latex tutorial and examples
%%%%%%%%%%%%%%%%%%%%%%%%%%%%%%%%%%%%%%%%%%%%%%%%%%%%%%%%%%%%%%%%%%%%

\typeout{NT FILE chapter3.tex}%


\chapter{Results}
\label{cha:results}

The algorithms seen in \ref{sec:Proposed_Decision_Algorithms}, namely, \label{subsec:decision_algorithms} \glsxtrshort{A-JO}, \glsxtrshort{A-CLF-S} and \glsxtrshort{A-CLF-I},  are going to be implemented in different systems, in a unicycle~\ref{subsec:unicycle_simul_setup} and orbital system~\ref{subsec:orbital_simul_setup}, and subject to different types of experiments according to the system in question (unicycle~\ref{subsubsec:unicyle_type_of_experiments} or orbital~\ref{subsubsec:orbital_type_of_experiments}). In each of the contexts, there is an analisys of the techniques and a comparation between them highlingting some relevant data.

\section{Setup}
\label{sec:setup}

\subsection{Decision Algorithms}
\label{subsec:decision_algorithms}

%The respective algorithms have some parameters to be defined to attend the problem needs.

\subsubsection{\glsxtrshort{A-JO}}
\label{subsubsec:A-JO_parameters}

The parameter \(\alpha\) is optimized based on a cost function \(g_{\xVec_0}(\alpha)\)~\ref{eq:Cost_Function_Alpha}, which takes \(N=6000\) samples and has the weight cost matrices \(\mathbf{Q} = \mathbf{I}_{2\times 2} \) and \(\mathbf{R} = 5 \hspace{0.1em}^.\mathbf{I}_{2\times 2} \).\\

The function \(SPS\)~\ref{eq:CLF-CBF_RK4_Propagation} and \(DGF\)~\ref{eq:discrete_gradient_function} parameters used in these experiments are given by:


 \bgroup
 \rowcolors{1}{}{GhostWhite}
 \begin{xltabular}{\textwidth}{ccccc}
   \caption{SPS~\ref{eq:CLF-CBF_RK4_Propagation} Parameters}
   \label{tab:A-JO:SPS_parameters}\\
   \toprule
   \rowcolor{Gainsboro}
   $f$ &  $\xVec_0$ & $N$ & $\Delta t$  & $V, \hspace{0.25em} h, \hspace{0.25em} \alpha_{ini}$  \\
   \midrule
     \ref{eq:unicycle_model} or \ref{}          &  $[0 \hspace{0.35em} 0]^{\top}$        & 6000          & 0.005  &   \ref{subsubsec:Unicyle_CLF-CBF_Experiment_Setup} or \ref{}\\
   \midrule
   \end{xltabular}
 \egroup




  \bgroup
 \rowcolors{1}{}{GhostWhite}
 \begin{xltabular}{\textwidth}{ccccc}
   \caption{DGF~\ref{eq:discrete_gradient_function} Parameters}
   \label{tab:A-JO:DGF_parameters}\\
   \toprule
   \rowcolor{Gainsboro}%
   $nS$ &  $devS$ & $dev$ & $g_{\xVec_0}(\alpha)$  & $\alpha_{ini}$  \\
   \midrule
     3          &  0.01        & 0.005        &  \ref{eq:Cost_Function_Alpha}   &   \ref{subsubsec:Unicyle_CLF-CBF_Experiment_Setup} or \ref{}\\
   \midrule
   \end{xltabular}
 \egroup


\newpage %so para ajeitar

 \subsubsection{Double \glsxtrshort{CLF}}
\label{subsubsec:double_CLF_parameters}

The double \glsxtrshort{CLF} algorithms~\ref{subsec:Double_CLF} functions parameters are defined by:

 \bgroup
 \rowcolors{1}{}{GhostWhite}
 \begin{xltabular}{\textwidth}{ccccc}
   \caption{NCDV~\ref{eq:New_Equlibrium_Point_DirVec_CLF-CBF_RK4} Parameters}
   \label{tab:Double-CLF:NCDV_parameters}\\
   \toprule
   \rowcolor{Gainsboro}%
   $f$ &  $\xVec_0$ & $N$ & $\Delta t$  & $V, \hspace{0.25em} h$  \\
   \midrule
     \ref{eq:unicycle_model} or \ref{}          &  $[0 \hspace{0.35em} 0]^{\top}$        & 6000          & 0.005  &   \ref{subsubsec:Unicyle_CLF-CBF_Experiment_Setup} or \ref{}\\
   \midrule
   \end{xltabular}
 \egroup


  \bgroup
 \rowcolors{1}{}{GhostWhite}
 \begin{xltabular}{\textwidth}{lcccc}
   \caption{Double \glsxtrshort{CLF}~\ref{subsec:Double_CLF} Transition Parameters}
   \label{tab:Double-CLF:NCDV_parameters}\\
   \toprule
   \rowcolor{Gainsboro}%
   Algorithms   & $\eta$  & $\mu$  & $tol$   & $tol_2$            \\
   \midrule
    \glsxtrshort{A-CLF-S}~\ref{subsubsec:CLFs_Summed_Algorithm} & 1       & $\frac{V_N(\xVec)}{V_C(\xVec)}$          & $0.03^.V_C(\xVec)$          & 999          \\
    \glsxtrshort{A-CLF-I}~\ref{subsubsec:CLFs_Independent_Algorithm}   & ---       & ---         & $0.05^.V_C(\xVec)$           & 50        \\
    \midrule
   \end{xltabular}
 \egroup



\subsection{Unicycle}
\label{subsec:unicycle_simul_setup}

This example consists in drive a unicycle to a specific location \(\mathbf{ref} \in \mathbb{R}^2\). Drive a vehicle implies the possibilitie of founding some obstacles, and while a collision could mean a fatality there are other factors as the fuel/energy consumed and how fast to achieve the wanted location, that beeing said, this represents an example of safety-critical control subject to an optimization of its path. \par

\subsubsection{Dynamics}
\label{subsec:unicycle_dynamics}

This unicycle doesn't loose traction in its wheels, and it is not constrained by its control inputs or its consume. It is also not affected by any perturbation such as drag force characteristic from the enviroment. It can be modeled mathematically as a relative two degree system:

\begin{equation}
    \begin{bmatrix} \dot{x} \\ \dot{y} \\ \dot{\theta} \end{bmatrix} = \begin{bmatrix} \frac{1}{2}cos \theta & \frac{1}{2}cos \theta \\  \frac{1}{2}sin \theta & \frac{1}{2}sin \theta \\ -L & L  \end{bmatrix} \begin{bmatrix} v_l \\ v_r\end{bmatrix}
    \label{eq:unicycle_model}
\end{equation}


Where the states are the \((x,y)\) 2D position and \(\theta\) the orientation, and the control inputs \(\uVec \in \mathbb{R}^2\) are the left and right wheel speed \(v_l\) and \(v_r\) respectively, in addition, the constant \(L = 10 m^{-1}\) refers to the inverse of the distance between the two wheels. 

\subsubsection{\glsxtrshort{CLF-CBF} Formulation}
\label{subsubsec:Unicyle_CLF-CBF_Experiment_Setup}

As the decision algoritms subject to comparations~\ref{subsec:decision_algorithms} are governed by the \glsxtrshort{CLF-CBF}~\ref{sec:clf_cbf} backstepping technique~\ref{sub:backstepping}, the system can be written as:


\begin{equation}
    \begin{array}{l}
        \dot{\xVec} = f_0(\xVec) +  g_0(\xVec, \xi)\uVec \\
        \dot{\xi} = f_1(\xVec, \xi) + g_1(\xVec, \xi)\uVec
    \end{array}
    \label{eq:second_order_unicycle_model_backstepp}
\end{equation}

With \(\xVec =  \bigl[\begin{smallmatrix} x \\ y \end{smallmatrix}\bigr]\), \(\xi =  \bigl[\begin{smallmatrix} cos\theta \\ sin\theta \end{smallmatrix}\bigr]\), \(f_0(\xVec) = \mathbf{0}_{2 \times 1}\), \(g_0(\xVec, \xi) = \bigl[\begin{smallmatrix} \frac{1}{2}\xi_1 & \frac{1}{2}\xi_1 \\ \frac{1}{2}\xi_2 & \frac{1}{2}\xi_2 \end{smallmatrix}\bigr] \), \(f_1(\xVec, \xi) = \mathbf{0}_{2 \times 1}\) and \(g_1(\xVec, \xi) =  \bigl[\begin{smallmatrix} \xi_2L & -\xi_2^L \\ -\xi_1L &  \xi_2L \end{smallmatrix}\bigr]\). \par

Since it's a second-order system, to use backstepping and obtain the first-order solution \(\mathbf{k}_0(\xVec)\) it is considered the system:

\begin{equation}
    \dot{\xVec} = k_0(\xVec) 
    \label{eq:first_order_unicycle_model_backstepp}
\end{equation}

In order to make the vehicle to the wanted position is formulated a \glsxtrshort{CLF} \(V_0\) which actively works to stabilize the system in the constant \(\mathbf{ref}\).

\begin{subequations}
   \begin{align}
    &V_0(\xVec) = \frac{1}{2}(\xVec-\mathbf{ref})^{\top}(\xVec-\mathbf{ref}) \label{eq:V0_unicycle} \\
    &\dot{V_0}(\xVec) = \underbrace{(\xVec-\mathbf{ref})^{\top}}_{L_GV_0(\xVec)}\mathbf{k}_0(\xVec)  \label{eq:dot_V0_unicycle} \\
    &\gamma_0(\xVec)  = V_0(\xVec) \label{eq:CLF_gamma0_unicycle}
\end{align}
\label{eq:CLF0_unicycle}
\end{subequations}


The safety is defined according to the \glsxtrshort{CBF} \(h_0\) which restricts the obstacles using a ellipsoidal fit~\ref{eq:Ellipsoidal_fit}:

\begin{subequations}
   \begin{align}
    &h_0(\xVec) = (\xVec-\mathbf{p})^{\top}\mathbf{A_{xis}}(\xVec-\mathbf{p}) - 1 \label{eq:h0_unicycle} \\
    &\dot{h_0}(\xVec) = (\underbrace{(\mathbf{A_{xis}}(\xVec-\mathbf{p}))^{\top} + (\xVec-\mathbf{p})^{\top}\mathbf{A_{xis}}}_{L_Gh_0(\xVec)})\mathbf{k}_0(\xVec)  \label{eq:dot_h0_unicycle} \\
    &\alpha_0(\xVec, \alpha)  = \alpha h_0(\xVec) \label{eq:CBF_alpha0_unicycle}
\end{align}
\label{eq:CBF0_unicycle}
\end{subequations}


With \(\mathbf{A_{xis}} \in \mathbb{R}^{2 \times 2}_{\succeq 0}\) referent to the ellipse axis and \(\mathbf{p} \in \mathbb{R}^2\) equal to the center of the ellipse, both stipulated in \ref{subsubsec:unicyle_type_of_experiments}. Given the system \ref{eq:second_order_unicycle_model_backstepp}, the solution \(\mathbf{k}_0(\xVec)\), the \glsxtrshort{CLF} \(V\) and \glsxtrshort{CBF} \(V\) are defined (according to \ref{eq:Backstepp_CF}) as:


\begin{subequations}
   \begin{align}
    &V(\xVec, \xi) = V_0(\xVec) + 1 - cos(\theta - \theta_0(\xVec)) \label{eq:V_unicycle} \\
    &\dot{V}(\xVec, \xi) = \Bigl(\underbrace{((\xVec-\mathbf{ref}))^{\top}g_0(\xVec, \mathbf{k}_{0,\xi}) + sin(\theta -\theta_0(\xVec))\begin{bmatrix} -L & L \end{bmatrix}}_{L_GV(\xVec, \xi)} \Bigr) \mathbf{k}(\xVec)  \label{eq:dot_V_unicycle} \\
    &\gamma(\xVec)  = V(\xVec) \label{eq:CLF_gamma_unicycle}
\end{align}
\label{eq:CLF_unicycle}
\end{subequations}

\begin{subequations}
   \begin{align}
    &h(\xVec, \xi) = h_0(\xVec) + 1 - cos(\theta - \theta_0(\xVec)) \label{eq:h_unicycle} \\
    &\dot{h}(\xVec, \xi) = \Bigl(\underbrace{\bigl((\mathbf{A_{xis}}(\xVec-\mathbf{p}))^{\top} + (\xVec-\mathbf{p})^{\top}\mathbf{A_{xis}}\bigr)g_0(\xVec, \mathbf{k}_{0,\xi}) + sin(\theta -\theta_0(\xVec))\begin{bmatrix} -L & L \end{bmatrix}}_{L_Gh(\xVec, \xi)} \Bigr) \mathbf{k}(\xVec)  \label{eq:dot_h_unicycle} \\
    &\alpha(\xVec, \xi, \alpha = 1.2)  = \alpha h(\xVec, \xi) \label{eq:CBF_alpha_unicycle}
\end{align}
\label{eq:CBF_unicycle}
\end{subequations}


With \(\mathbf{k}_{0,\xi}(\xVec)\) and \(\theta(\xVec)\) given by:

\begin{align}
    &\mathbf{k}_{0,\xi}(\xVec) = \frac{\mathbf{k}_0(\xVec) }{|| \mathbf{k}_0(\xVec) ||}      \\
    &\theta_0(\xVec) = angle(\mathbf{k}_{0,\xi,1}(\xVec) + \mathbf{i} \hspace{0.2em} \mathbf{k}_{0,\xi,2}(\xVec))
\end{align}


\subsubsection{Type of Experiments}
\label{subsubsec:unicyle_type_of_experiments}

-falar do tipo de simulaçoes que se vai fazer\\
-falar dos valores tomados pelas axis das elipses


\subsection{Orbital}
\label{subsec:orbital_simul_setup}

\subsubsection{Dynamics}
\label{subsubsec:orbital_dynamics}

-apresentar modelo orbital\\
-CLF e CBF deste modelo e referindo o facto do fit ser elipsoidal

\subsubsection{\glsxtrshort{CLF-CBF} Formulation}
\label{subsubsec:Orbital_CLF-CBF_Experiment_Setup}


\subsubsection{Type of Experiments}
\label{subsubsec:orbital_type_of_experiments}

-falar do tipo de simulaçoes que se vai fazer\\
-falar dos valores tomados pelas axis das elipses


\subsection{Some Data}
\label{subsec:some_data}

The techniques will be evaluated having in account a cost, given by the function \(g(\mathbf{X}, \mathbf{U})\), equivalent to \(g_{\bar{\xVec_0}}(\alpha)\) defined before~\ref{subsubsec:A-JO_parameters}, such as the weight cost matrices, \(\mathbf{Q} = \mathbf{I}_{2\times 2} \) and \(\mathbf{R} = 5 \hspace{0.1em}^.\mathbf{I}_{2\times 2} \), besides that the function is calculated with \(N=6000\) samples. It is also presented the time the system \(f\) takes to reach the equilibrium point.\par
The computational time for each experiment is given based on a single test (one sample Monte Carlo) using the the processor \emph{Intel(R) Core(TM) i7-8650U CPU @ 1.90GHz}.





\section{Experiments}
\label{sec:experiments}

The dynamical system propagation with respective decision algorithm~\ref{subsec:decision_algorithms} during the experiments is simulated using the \glsxtrlong{DP} integrative method~\ref{eq:Dormand-Prince_Tableu}. So, given a dynamical system (unicycle~\ref{eq:unicycle_model} or orbital~\ref{subsubsec:orbital_model}) \(f:\mathbb{R}^n \to \mathbb{R}^n\), at \( \xVec_0 \in \mathbb{R}^n\), a \glsxtrshort{CLF} \(V\) and a \glsxtrshort{CBF} \(h\), a sampling time \(\Delta t = 0.005\) seconds and a total of \(6000\) iterations totaling a simulation of \(30\) seconds:


\begin{equation}
    \begin{bmatrix} \mathbf{X} & \mathbf{U} \end{bmatrix} = SPS(f, \xVec_0, 6000, 0.005, V, h)\begin{bmatrix*}[l]conditions* \gets V(\xVec) \geq tol \\ decision^* \gets \text{algorithms~\ref{subsec:decision_algorithms}} \\ integration^* \gets \text{\glsxtrshort{DP}~\ref{eq:Dormand-Prince_Tableu}}\end{bmatrix*}
    \label{eq:SPS_Experiments}
\end{equation}

Where \(SPS\) (State Propagation Simulation) is the algorithm~\ref{alg:State_Propagation_Simulation}.\\



% \subsection{Unicycle Model}
% \label{subsec:unicycle_model}

% \subsubsection{Circular Constraint}
% \label{subsubsec:Circular_Constraint}

% \subsubsection{Ellipsoidal Constraint}
% \label{subsubsec:Ellipsoidal_Constraint}

% \subsubsection{Double Ellipsoidal Constraint}
% \label{subsubsec:Double_Ellipsoidal_Constraint}

% \subsection{Orbital Model}
% \label{subsec:orbital_model}

% \subsubsection{Circular Constraint}
% \label{subsubsec:Circular_Constraint}

% \subsubsection{Ellipsoidal Constraint}
% \label{subsubsec:Ellipsoidal_Constraint}

% \subsubsection{Double Ellipsoidal Constraint}
% \label{subsubsec:Double_Ellipsoidal_Constraint}


-mostrar as simulaçoes\\
-explicar os resultados obtidos

















% Most latex users will instead create more and more advanced commands.
% A lot of us have a "commands.tex" that we "\input{commands.text}" at the beginning of all our documents.
% Taking simple examples, I have \newcommand{\CC}{\mathbb{C}} at part of the commands in most of my documents because I use the field of complex number everywhere and \CC is shorter to write than \mathbb{C}. I also have \newenvironment{ctabular}[1]{\begin{center}\begin{tabular}{#1}}{\end{tabular}\end{center}} for whenever I need a centred tabular, which is almost always.

% \makeatletter %Transforma o @ para um caracter normal
% \newcommand{\ntifpkgloaded}{%
%   \@ifpackageloaded% %Não esta ai mas seria mais 3 argumentos {}{}{} o primeiro o package, o segundo e o terceiro if e if not o package tiver sido load respetivamente
% }
% \makeatother %Transforma o @ para um caracter especial outra vez

% \lipsum[1-3]

% \begin{figure}[htbp]
%   \centering
%   \subbottom[One sub-figure\label{fig:leftsubfig}]{%
%     \includegraphics[width=0.5\linewidth]{knitting-vectorial}}%
%   \subbottom[Another sub-figure\label{fig:rightsubfig}]{%
%     \includegraphics[width=0.5\linewidth]{knitting-vectorial}}%
%   \caption{A figure with two sub-figures!}
%   \label{fig:fig2subfig}
% \end{figure}



%  The Table~\ref{tab:hla:results} illustrates some important concepts associated with table construction:
%  \begin{asparaenum}[i)]
%  \item Do not use vertical lines;
%  \item The caption should be above the table;
%  \item Use the macros \verb!\toprule!, \verb!\midrule! and \verb!\bottomrule! to make the top, inner and bottom horizontal lines, respectively.
%  \end{asparaenum}

%  \bgroup
%  \rowcolors{1}{}{GhostWhite}
%  \begin{xltabular}{\textwidth}{Xccccc}
%    \caption{Test results summary.}
%    \label{tab:hla:results}\\
%    \toprule
%    \rowcolor{Gainsboro}%
%    Test   & Anomalies  & Warnings  & Correct   & Categories            & Missed \\
%    \midrule
%  Connection~\cite{Beckman08}     & 2       & 2          & 1          & \emph{C}              & 1 \\
%  Coordinates'03~\cite{Artho03}   & 1        & 4          & 1          & \emph{2B, 1C}          & 0 \\
%  Local Variable~\cite{Artho03}    & 1        & 2          & 1          & \emph{A}              & 0 \\
%  NASA~\cite{Artho03}              & 1        & 1          & 1          & ---                    & 0 \\
%    \midrule
%    \rowcolor{Gainsboro}%
%  Total                            & 12      & 33        & 10        & 5A, 6B, 10C, 2D       & 2 \\
%    \bottomrule
%    \end{xltabular}
%  \egroup


% \begin{figure}[htbp]
%   \centering
%   \includegraphics[height=1in]{snowman-vectorial}
%   \includegraphics[height=3in]{snowman-vectorial}
%   \includegraphics[height=6in]{snowman-vectorial}
%   \caption{Vectorial image (PDF)}
%   \label{fig:Figures_Tree_silhouettes-vectorial}
% \end{figure}

% To combine several figures into a single one… You can then reference the set as Figure~\ref{fig:complete-figure} or the sub-figures separately as~\ref{fig:woolball} and~\ref{fig:cloud}.

% \begin{figure}[htbp]
%   \centering
%     \subbottom[Novelo de lã]{%
%     \label{fig:woolball}
%     \includegraphics[height=1in]{knitting-vectorial}
%     }
% \qquad\qquad
%     \subbottom[Tempestade com neve]{%
%     \label{fig:cloud}
%     \includegraphics[height=1in]{snowstorm-vectorial}
%     }
%   \caption{Exemplo de utilização de \emph{subbottom}}
%   \label{fig:complete-figure}
% \end{figure}


% Uncomment the algorithms source below and add the following to file “\verb!5_packages.tex!”
% \begin{verbatim}
%   \usepackage{algorithm2e}
%   \RestyleAlgo{ruled}
% \end{verbatim}
% and uncomment
% \begin{verbatim}
% \ntaddlistof{listofalgorithms}
% \end{verbatim}
% in file “\verb!8_list_og.tex!”.



%NÂO CONSEGUI POR A FUNCIONAR MAS OLHEI POUCO

% \newif\ifntlistingsloaded
% \ntifpkgloaded{listings}{\ntlistingsloadedtrue}{\ntlistingsloadedfalse}
% \newif\ifntmintedloaded
% \ntifpkgloaded{minted}{\ntmintedloadedtrue}{\ntmintedloadedfalse}

% \ifntlistingsloaded
% \section{Test for listings} % (fold)
% \label{sec:test_for_listings}

% Testing the package “listings“…

% \begin{lstlisting}[caption=cap,label=lst:lab,float=htbp]
% if(a==b)
%   puts("YESS!")
% \end{lstlisting}
% \fi

% \ifntmintedloaded
% \section{Test for minted} % (fold)
% \label{sec:test_for_minted}

% Testing the package “minted“…

% \begin{listing}[H]
%   \begin{minted}{C}
%     if(a==b)
%       puts("YESS!")
%   \end{minted}
%   \caption{Example of a listing.}
%   \label{lst:lab}
% \end{listing}
% \fi
