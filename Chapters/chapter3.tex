%!TEX root = ../template.tex
%%%%%%%%%%%%%%%%%%%%%%%%%%%%%%%%%%%%%%%%%%%%%%%%%%%%%%%%%%%%%%%%%%%%
%% chapter3.tex
%% NOVA thesis document file
%%
%% Chapter with a short latex tutorial and examples
%%%%%%%%%%%%%%%%%%%%%%%%%%%%%%%%%%%%%%%%%%%%%%%%%%%%%%%%%%%%%%%%%%%%

\typeout{NT FILE chapter3.tex}%


\chapter{Results}
\label{cha:results}
























% Most latex users will instead create more and more advanced commands.
% A lot of us have a "commands.tex" that we "\input{commands.text}" at the beginning of all our documents.
% Taking simple examples, I have \newcommand{\CC}{\mathbb{C}} at part of the commands in most of my documents because I use the field of complex number everywhere and \CC is shorter to write than \mathbb{C}. I also have \newenvironment{ctabular}[1]{\begin{center}\begin{tabular}{#1}}{\end{tabular}\end{center}} for whenever I need a centred tabular, which is almost always.

% \makeatletter %Transforma o @ para um caracter normal
% \newcommand{\ntifpkgloaded}{%
%   \@ifpackageloaded% %Não esta ai mas seria mais 3 argumentos {}{}{} o primeiro o package, o segundo e o terceiro if e if not o package tiver sido load respetivamente
% }
% \makeatother %Transforma o @ para um caracter especial outra vez

% \lipsum[1-3]

% \begin{figure}[htbp]
%   \centering
%   \subbottom[One sub-figure\label{fig:leftsubfig}]{%
%     \includegraphics[width=0.5\linewidth]{knitting-vectorial}}%
%   \subbottom[Another sub-figure\label{fig:rightsubfig}]{%
%     \includegraphics[width=0.5\linewidth]{knitting-vectorial}}%
%   \caption{A figure with two sub-figures!}
%   \label{fig:fig2subfig}
% \end{figure}

% \textbf{And this is a small text that references the Figure~\ref{fig:fig2subfig} and its Subfigures~\ref{fig:leftsubfig} and~\ref{fig:rightsubfig}.}

% Footnotes\footnote{This is a simple footnote.} will be numbered and shown in the bottom of the page.


% The Table~\ref{tab:hla:results} illustrates some important concepts associated with table construction:
% \begin{asparaenum}[i)]
% \item Do not use vertical lines;
% \item The caption should be above the table;
% \item Use the macros \verb!\toprule!, \verb!\midrule! and \verb!\bottomrule! to make the top, inner and bottom horizontal lines, respectively.
% \end{asparaenum}

% \bgroup
% \rowcolors{1}{}{GhostWhite}
% \begin{xltabular}{\textwidth}{Xccccc}
%   \caption{Test results summary.}
%   \label{tab:hla:results}\\
%   \toprule
%   \rowcolor{Gainsboro}%
%   Test   & Anomalies  & Warnings  & Correct   & Categories            & Missed \\
%   \midrule
% Connection~\cite{Beckman08}     & 2       & 2          & 1          & \emph{C}              & 1 \\
% Coordinates'03~\cite{Artho03}   & 1        & 4          & 1          & \emph{2B, 1C}          & 0 \\
% Local Variable~\cite{Artho03}    & 1        & 2          & 1          & \emph{A}              & 0 \\
% NASA~\cite{Artho03}              & 1        & 1          & 1          & ---                    & 0 \\
%   \midrule
%   \rowcolor{Gainsboro}%
% Total                            & 12      & 33        & 10        & 5A, 6B, 10C, 2D       & 2 \\
%   \bottomrule
%   \end{xltabular}
% \egroup


% \begin{figure}[htbp]
%   \centering
%   \includegraphics[height=1in]{snowman-vectorial}
%   \includegraphics[height=3in]{snowman-vectorial}
%   \includegraphics[height=6in]{snowman-vectorial}
%   \caption{Vectorial image (PDF)}
%   \label{fig:Figures_Tree_silhouettes-vectorial}
% \end{figure}

% To combine several figures into a single one… You can then reference the set as Figure~\ref{fig:complete-figure} or the sub-figures separately as~\ref{fig:woolball} and~\ref{fig:cloud}.

% \begin{figure}[htbp]
%   \centering
%     \subbottom[Novelo de lã]{%
%     \label{fig:woolball}
%     \includegraphics[height=1in]{knitting-vectorial}
%     }
% \qquad\qquad
%     \subbottom[Tempestade com neve]{%
%     \label{fig:cloud}
%     \includegraphics[height=1in]{snowstorm-vectorial}
%     }
%   \caption{Exemplo de utilização de \emph{subbottom}}
%   \label{fig:complete-figure}
% \end{figure}

% LaTeX is a powerful tool for writing in a mathematical style. It allows you to insert formulas into the text, such as this: $ax^2 + bx + c = 0$. It also allows formulas to be highlighted on a separate line and centered on the page.
% \[ x = \frac{-b \pm \sqrt{b^2-4ac}}{2a} \]
% or numbererd
% \begin{equation}
% e = mc^2
% \label{eq:1}
% \end{equation}
% which can latter be referenced as equation~\ref{eq:1}



% Uncomment the algorithms source below and add the following to file “\verb!5_packages.tex!”
% \begin{verbatim}
%   \usepackage{algorithm2e}
%   \RestyleAlgo{ruled}
% \end{verbatim}
% and uncomment
% \begin{verbatim}
% \ntaddlistof{listofalgorithms}
% \end{verbatim}
% in file “\verb!8_list_og.tex!”.

%  \begin{algorithm}
%  $i\gets 10$\;
%  \eIf{$i\geq 5$}
%  {
%      $i\gets i-1$\;
%  }{
%      \If{$i\leq 3$}
%      {
%          $i\gets i+2$\;
%      }
%  }
% \caption{This is an algorithm.}
%  \end{algorithm}




%NÂO CONSEGUI POR A FUNCIONAR MAS OLHEI POUCO

% \newif\ifntlistingsloaded
% \ntifpkgloaded{listings}{\ntlistingsloadedtrue}{\ntlistingsloadedfalse}
% \newif\ifntmintedloaded
% \ntifpkgloaded{minted}{\ntmintedloadedtrue}{\ntmintedloadedfalse}

% \ifntlistingsloaded
% \section{Test for listings} % (fold)
% \label{sec:test_for_listings}

% Testing the package “listings“…

% \begin{lstlisting}[caption=cap,label=lst:lab,float=htbp]
% if(a==b)
%   puts("YESS!")
% \end{lstlisting}
% \fi

% \ifntmintedloaded
% \section{Test for minted} % (fold)
% \label{sec:test_for_minted}

% Testing the package “minted“…

% \begin{listing}[H]
%   \begin{minted}{C}
%     if(a==b)
%       puts("YESS!")
%   \end{minted}
%   \caption{Example of a listing.}
%   \label{lst:lab}
% \end{listing}
% \fi
