%!TEX root = ../template.tex
%%%%%%%%%%%%%%%%%%%%%%%%%%%%%%%%%%%%%%%%%%%%%%%%%%%%%%%%%%%%%%%%%%%%
%% chapter2.tex
%% NOVA thesis document file
%%
%% Chapter with the template manual
%%%%%%%%%%%%%%%%%%%%%%%%%%%%%%%%%%%%%%%%%%%%%%%%%%%%%%%%%%%%%%%%%%%%

\typeout{NT FILE chapter2.tex}%


\chapter{Methodology}
\label{cha:methodology}

\glsresetall %Para em vez de mostrar o acrónimo voltar a mostrar o nome completo para relembrar



%Aqui falar do modelo generico e de algumas coisas que serviram de base para os proximos topicos e que não valem a pena uma secção, alem de introduzir o que ai vem

\section{\glsxtrshort{CLF-CBF}}
\label{sec:clf_cbf}


\subsection{1st Order Formulation}
\label{sub:formulation}

\subsubsection{\glsxtrfull{CLF}}
\label{subsub:control_lyapunov_function}

Having the objective of global assymptotically stability of the system~\ref{eq:}, is desired for the system converge to an equilibrium point \( \bar{\mathbf{x}} \). A solution comes by designing a positive definite function \( V: \mathbb{R}^n \rightarrow \mathbb{R}_{\geq 0}  \) which converges to zero while the system converges to the respective equilibrium point, and for that, designing a control law that acts in order to achieve specifically the convergence of the function. Consequentially is pretended to the function \( V\mathbf(x) \) to be continuos and differentiable:

\[\dot{V} (\mathbf{x}, \mathbf{u}) = \frac{\partial V}{\partial x}\dot{x} = L_fV(\mathbf{x}) + L_GV(\mathbf{x})\mathbf{u} \]

where \(L_fV\) and \(L_GV\) are the Lie derivatives of \(V\) and \(\mathbf{u}\) the control input vector resposible to manipulate the function, more precisely, its convergence rate.\par

That beeing said, a continuosly differentiable, positive definite function \( V: \mathbb{R}^n \rightarrow \mathbb{R}_{\geq 0}  \) wich \( \lim_{\mathbf{x} \to \bar{\mathbf{x}}}{V(\mathbf{x})} = 0 \) is a \glsxtrshort{CLF} if there exists a \( \gamma \in \mathcal{K}^e_{\infty}  \)\footnote{METER AQUI UMA MENSAGEM A RODAPE A EXPLICAR ISTO}  such that for all \( \mathbf{x} \in \mathbb{R}^n \setminus \{\bar{\mathbf{x}}\} \):

\begin{equation}
 \inf_{\mathbf{u} \in \mathbb{R}^m} [L_fV(\mathbf{x}) + L_GV(\mathbf{x})\mathbf{u}] \leq -\gamma(V(\mathbf{x}))
 \label{eq:CLF}
\end{equation}

From \glsxtrshort{CLF} \(V\) and an adequate tuning of \( \gamma \) parameter results a control law followed by the controller \(K_{CLF}\):

\begin{equation}
 K_{CLF}(\mathbf{x}) = \{ \mathbf{u} \in \mathbb{R}^m: L_fV(\mathbf{x}) + L_GV(\mathbf{x})\mathbf{u} \leq -\gamma(V(\mathbf{x})) \}
 \label{eq:K-CLF}
\end{equation}

The stets generated by \(K_{CLF}(\mathbf{x}) \hspace{0.5em} \forall x \in \mathbb{R}^n \) refer to the possible control input values necessary to the \glsxtrshort{CLF} \(V(\mathbf{x})\) continuosly converge to zero while the closed-loop system~\ref{} to the equilibrium point, but more slowly as is close to reach the objective.


\subsubsection{\glsxtrfull{CBF}}
\label{subsub:control_barrier_function}

Now having the objective of garantee safety to system~\ref{}, is aimed to keep the state inside a safe set \(\mathcal{C}\) (foward invariant\footnote{METER AQUI EXPLICAÇÂO}) defined accordingly the problem needs. The notion of safety, i.e., the safe set \(\mathcal{C}\) is defined as the 0-superlevel\footnote{Explicação} set of a continuosly differentiable function \( h: \mathbb{R}^n \rightarrow \mathbb{R}  \):

\begin{equation}
    \begin{array}{l}
        \qquad   \mathcal{C} = \{ \mathbf{x} \in \mathbb{R}^n : h( \mathbf{x} )\geq 0\} \\
        \qquad \hspace{-0.5em}  \partial\mathcal{C} = \{ \mathbf{x} \in \mathbb{R}^n : h( \mathbf{x} ) = 0\} \\
        int(\mathcal{C}) = \{ \mathbf{x} \in \mathbb{R}^n : h( \mathbf{x} ) > 0\}
    \end{array}
 \label{eq:safe-set}
\end{equation}


In order to mantain \(h( \mathbf{x} )\) positive, there is a need to have a control law and but first to be able to interface with the function which is the case since \(h( \mathbf{x} )\) is differentiable:  

\[\dot{h}(\mathbf{x}, \mathbf{u}) = \frac{\partial h}{\partial x}\dot{x} = L_fh(\mathbf{x}) + L_Gh(\mathbf{x})\mathbf{u} \]

where \(L_fh\) and \(L_Gh\) are the Lie derivatives of \(h\) and \(\mathbf{u}\) the control input vector resposible to manipulate the function, more precisely, its time derivative.\par

So, a differentiable function \( h: \mathbb{R}^n \rightarrow \mathbb{R}  \) is a \glsxtrshort{CBF} if there exists a \( \alpha \in \mathcal{K}^e_{\infty}  \) such that for all \( \mathbf{x} \in \mathbb{R}^n \setminus \{\bar{\mathbf{x}}\} \):

\begin{equation}
 \sup_{\mathbf{u} \in \mathbb{R}^m} [L_fh(\mathbf{x}) + L_Gh(\mathbf{x})\mathbf{u}] \geq -\alpha(h( \mathbf{x} ))
 \label{eq:CBF}
\end{equation}

Similar to the \glsxtrshort{CLF}, from \glsxtrshort{CBF} \(h\) and an adequate tuning of \( \alpha \) parameter results a control law encapsulated by the controller \(K_{CBF}\):

\begin{equation}
 K_{CBF}(\mathbf{x}) = \{ \mathbf{u} \in \mathbb{R}^m: L_fh(\mathbf{x}) + L_Gh(\mathbf{x})\mathbf{u} \geq -\alpha(h( \mathbf{x} )) \}
 \label{eq:K-CBF}
\end{equation}


The controller \(K_{CLF}(\mathbf{x}) \hspace{0.5em} \forall x \in \mathbb{R}^n \) renders control input sets that keep the closed-loop system safe with respect to the set \(\mathcal{C}\). The controller by defenition is able to evolve the closed-loop system~\ref{} more freely when \(\dot{h}( \mathbf{x} ) > 0\) and more cautiously as \(h( \mathbf{x} ) \) shortens and \(\dot{h}( \mathbf{x} ) < 0\), if \(h( \mathbf{x} ) < 0\) it means is in the unsafe region so \(\dot{h}( \mathbf{x}) > 0\) converging continuosly to the safe set \(\mathcal{C}\). 




\subsubsection{Quadratic Program Formulation}
\label{subsub:quadratic_program_formulation}

Now having the objective of not only achieve safety but also assymptotically stability through the usage of \glsxtrshort{CLF} and \glsxtrshort{CBF} it renders a set of control inputs able to achieve the respective purposes.  

Using the conditions imposed by \glsxtrshort{CLF} and \glsxtrshort{CBF} as constraints, is presented an optimization problem  via \glsxtrfull{QP}\footnote{Explicar o que é um QP genericamente}, given a locally Lipschitz continuos\footnote{Outra explicação AQUI} controller \(k: \mathbb{R}^n \rightarrow \mathbb{R}^m \):

\begin{equation}
    \begin{array}{l}
        k(\xVec) = \underset{\uVec \in \mathbb{R}^m }{\mathrm{argmin}} \quad \hspace{-0.5em} \frac{1}{2} (\uVec^\top R \uVec) \vspace{1em} \\  
        \quad \hspace{0.5em}  st. \hspace{0.5em} L_fV(\mathbf{x}) + L_GV(\mathbf{x})\mathbf{u} \leq -\gamma(V(\mathbf{x})) \vspace{0.25em} \\ 
        \quad \quad \hspace{1em} L_fh(\mathbf{x}) + L_Gh(\mathbf{x})\mathbf{u} \geq -\alpha(h(\mathbf{x}))
    \end{array}
 \label{eq:CLF-CBF-QP}
\end{equation}

The sets of control inputs restricted to the optimization problem are given by \(K_{CLF-CBF}(\xVec) = K_{CLF}(\xVec) \cap  K_{CBF}(\xVec)\) hopefully non-empty which can happen if the conditions are in conflict. \par

In case of conditions' conflict, in order to the closed loop system~\ref{} controller have a solution from the \glsxtrshort{QP}, since safety is critical, using a slack variable \(\delta\)  allows to relax the stability condition imposed by the \glsxtrshort{CLF} and highly contribute to obtaining a solution while keeping system safe:

\begin{equation}
    \begin{array}{l}
        k(\xVec) = \underset{\uVec \in \mathbb{R}^m, \hspace{0.2em} \delta \in \mathbb{R}}{\mathrm{argmin}} \quad \hspace{-0.5em} \frac{1}{2} (\uVec^\top R \uVec + p\delta^2 ) \vspace{1em} \\ 
        \quad \hspace{0.5em}  st. \hspace{0.5em} L_fV(\mathbf{x}) + L_GV(\mathbf{x})\mathbf{u} \leq -\gamma(V(\mathbf{x})) + \delta \vspace{0.25em}\\
        \quad \quad \hspace{1em} L_fh(\mathbf{x}) + L_Gh(\mathbf{x})\mathbf{u} \geq -\alpha(h(\mathbf{x}))
    \end{array}
 \label{eq:CLF-CBF-QP-relaxed}
\end{equation}

Having \(\delta\) as an optimization variable not only allows to balance between the input values exceed usage and the level of relaxation but also to not have a fixed \(\delta\) value which would result for all states not so faster convergence and to the neighborhood of the equilibrium point instead of the actual target point. 

%tikz daquele esquema

%fazer caixinha azul dizendo que se pode retirar as respetivas constraints se não se pretende com o controlador obtido ter safety ou stability. E que o p\delta costuma ser com os inputs e p e uma triz quadrada de largira e comprimento igual ao numero de CLFs mas so se esta a assumir pois ao se obter closed-form o fazer permite uma soluçao mais simples e visto que a principio ter mais que uma CLF implica ter defenicoes de establidades apartir de certo ponto paradoxais ou que podem entrar em conflito para o sistema, relativamente as CBF, pode as unificar obtendo uma unica CBF como se pode ver na sua secção

\subsection{Backstepping}
\label{sub:backstepping}

\subsubsection{Higher Order Systems}
\label{subsub:higher_order_systems}

Usually control systems are governed by higher-order systems, such as some seconder-order cyber-physical systems controlling the position by applying some force to manipulate the speed that is felt in the position just an instant after.  Higher order systems' outputs variation is felt by an multiple integrative effect, by an integration of each state affecting the next state dynamic until reach the so wanted output. Considering a non-linear system in strict-feedback form:

% \begin{equation}
%     \begin{array}{l}
%         \dot{\xVec} = f_0(\xVec) + g_0(\xVec) \mathbf{\xi}\\ 
%         \dot{ \mathbf{\xi} } =  f_1(\xVec, \mathbf{\xi}) + g_1(\xVec, \mathbf{\xi}) \uVec
%     \end{array}
%  \label{eq:HH-NL-System}
% \end{equation}

\begin{subequations}
    
    \label{eq:HH-NL-System}
    \begin{align}
        \dot{\xVec} = f_0(\xVec) + g_0(\xVec) \mathbf{\xi} 
        \label{eq:HH-NL-System-1stOrder}\\
        \dot{ \mathbf{\xi} } =  f_1(\xVec, \mathbf{\xi}) + g_1(\xVec, \mathbf{\xi}) \uVec
        \label{eq:HH-NL-System-2ndOrder}
    \end{align}
\end{subequations}

with \(\mathbf{x} \in \mathbb{R}^n \), \(\mathbf{\xi} \in \mathbb{R}^p \), and \(\mathbf{u} \in \mathbb{R}^m \), and functions \(f_0: \mathbb{R}^n \to \mathbb{R}^n \), \(g_0: \mathbb{R}^n \to \mathbb{R}^{n \times p} \), \(f_1: \mathbb{R}^n \times \mathbb{R}^p \to \mathbb{R}^p \) and \(g_1: \mathbb{R}^n \times \mathbb{R}^p \to \mathbb{R}^{ p \times m} \) locally Lipschitz continuos on their respective domains. \par
Given a locally Lipschitz continuos feedback controller \(\mathbf{k}: \mathbb{R}^n \times \mathbb{R}^p \to \mathbb{R}^m  \) the closed-loop system is:

\begin{equation}
    \begin{array}{l}
        \dot{\xVec} = f_0(\xVec) + g_0(\xVec) \mathbf{\xi}\\ 
        \dot{ \mathbf{\xi} } =  f_1(\xVec, \mathbf{\xi}) + g_1(\xVec, \mathbf{\xi}) \mathbf{k}(\xVec, \mathbf{\xi})
    \end{array}
 \label{eq:HH-NL-CL-System}
\end{equation}

for any initial states \((\xVec_0, \mathbf{\xi}_0) \in \mathbb{R}^n \times \mathbb{R}^p\) exists a maximum time interval \(I(\xVec_0, \mathbf{\xi}_0) \subseteq \mathbb{R}_{\geq 0}  \) and a unique solution \( \varphi = (\varphi_{\mathbf{x}}, \varphi_{\mathbf{\xi}}) \) satisfying \ref{} \(\forall t \in I(\xVec_0, \mathbf{\xi}_0) \). \\


\subsubsection{Purposal  Solution Obtaining}
\label{subsub:quadratic_program_formulation}

Since having a second order system~\ref{eq:HH-NL-System} the access to \(\mathbf{\xi}\) is not direct, we must "backstepp" in order to be able to control \(\uVec\), affect \(\mathbf{\xi}\) and consequentially control the wanted dynamic \(\dot{\xVec}\). The strategy passes by obtaining the solution if it was a first order system~\ref{} and if \(\mathbf{\xi}\) could be directly controlled (i.e. as part of the first order system control input), so given a differentiable and locally Lipschitz continuos \(k_0(\xVec):\mathbb{R}^n \to \mathbb{R}^p\) (that encompasses the "controllable \(\mathbf{\xi}\)" ), the closed-loop system:

\begin{equation}
    \dot{\xVec} = f_0(\xVec) + g_0(\xVec) k_0(\xVec)\\ 
 \label{eq:NL-CL-System-0Backstep}
\end{equation}

That beeing said, the first order solution could be obtained via \glsxtrshort{QP}, so supposing a \glsxtrshort{CLF} \(V_0\),  \glsxtrshort{CBF} \(h_0\),  a differentiable and locally Lipschitz continuos described just now \(k_0(\xVec)\) and \((\gamma_0, \alpha_0) \in \mathcal{K}^e_\infty \times \mathcal{K}^e_\infty\):

\begin{equation}
    \begin{array}{l}
        k_0(\xVec) = \underset{\uVec \in \mathbb{R}^p, \hspace{0.2em} \delta \in \mathbb{R}}{\mathrm{argmin}} \quad \hspace{-0.5em} \frac{1}{2} (\uVec^\top R_0 \uVec + p_0\delta^2 ) \vspace{1em} \\ 
        \quad \hspace{0.5em}  st. \hspace{0.5em} L_fV_0(\mathbf{x}) + L_GV_0(\mathbf{x})\mathbf{u} \leq -\gamma_0(V_0(\mathbf{x})) + \delta \vspace{0.25em}\\
        \quad \quad \hspace{1em} L_fh_0(\mathbf{x}) + L_Gh_0(\mathbf{x})\mathbf{u} \geq -\alpha_0(h_0(\mathbf{x}))
    \end{array}
 \label{eq:CLF-CBF-QP-relaxed-0Backstep}
\end{equation}


The solution is equivalent to a value of \(\mathbf{\xi}\) closest to an optimal value taken by it that would impose safety and contribute to stability not only to the first-order system~\ref{eq:NL-CL-System-0Backstep} but also to the actual second-order system. Although like it was said is not possible to have direct access to \(\mathbf{\xi}\), the backstepping technique is essentially the second-order trying to converge to the first order solution. Supposing there is \( V_{\mathbf{\xi}}(\xVec, \mathbf{\xi}): \mathbb{R}^n \times \mathbb{R}^p \rightarrow \mathbb{R}_{\geq 0} \), the new \glsxtrshort{CLF} \(V(\xVec, \mathbf{\xi})\) and \glsxtrshort{CBF} \(h(\xVec, \mathbf{\xi})\) are:

\begin{align*}
    V_{\mathbf{\xi}}(\xVec, \mathbf{\xi}) = \frac{1}{2}|| \mathbf{\xi} -  k_0(\xVec) ||^2 \\
    V(\xVec, \mathbf{\xi}) = V_0(\xVec) + V_{\mathbf{\xi}}(\xVec, \mathbf{\xi}) \\
    h(\xVec, \mathbf{\xi}) = h_0(\xVec) - V_{\mathbf{\xi}}(\xVec, \mathbf{\xi})
\end{align*}

Finally, via \glsxtrshort{QP}, given a differentiable and locally Lipschitz continuos \(k(\xVec, \mathbf{\xi}):\mathbb{R}^n \times \mathbb{R}^p \to \mathbb{R}^m\):

\begin{equation}
    \begin{array}{l}
        k(\xVec, \mathbf{\xi}) = \underset{\uVec \in \mathbb{R}^p, \hspace{0.2em} \delta \in \mathbb{R}}{\mathrm{argmin}} \quad \hspace{-0.5em} \frac{1}{2} (\uVec^\top R \uVec + p\delta^2 ) \vspace{1em} \\ 
        \quad \hspace{0.5em}  st. \hspace{0.5em} L_fV(\xVec, \mathbf{\xi}) + L_GV(\xVec, \mathbf{\xi})\mathbf{u} \leq -\gamma(V_0(\mathbf{x})) -\gamma'(V_{\mathbf{\xi}}(\mathbf{x}, \mathbf{\xi})) + \delta \vspace{0.25em}\\
        \quad \quad \hspace{1em}                 L_fh(\xVec, \mathbf{\xi}) + L_Gh(\xVec, \mathbf{\xi})\mathbf{u} \geq -\alpha(h_0(\mathbf{x})) + \alpha'(V_{\mathbf{\xi}}(\mathbf{x}, \mathbf{\xi}))
    \end{array}
 \label{eq:CLF-CBF-QP-relaxed-1Backstep}
\end{equation}

It's expected that  \( \gamma << \gamma'\)  contributing to a faster convergence of \(\xi\) and more similar behaviour of the system~\ref{eq:HH-NL-System} relative to the first-order system~\ref{eq:HH-NL-System-1stOrder} and mitigate the chances of the system~\ref{eq:HH-NL-System} not be safe in relation to \(\mathcal{C}_0\).
%na ciaxinha a azul desta vez alem de falar que se nao se pretende safety ou stability continua a passar não usar as respetivas constraints. Mas agora falar caso fosse ordens maiores era pegar nesta ultima solução e assumir como se agora fosse o novo k0, fazendo "backstepp" ate à higher degree dynamic que influencia o output. 

% \begin{equation}
%     \begin{array}{l}
%         k_0(\xVec) = \underset{\uVec \in \mathbb{R}^p, \hspace{0.2em} \delta \in \mathbb{R}}{\mathrm{argmin}} \quad \hspace{-0.5em} \frac{1}{2} (\uVec^\top R \uVec + p\delta^2 ) \vspace{1em} \\ 
%         \quad \hspace{0.5em}  st. \hspace{0.5em} \underbrace{L_fV_0(\mathbf{x}) + L_GV_0(\mathbf{x})\mathbf{\xi} + L_fV_{\mathbf{\xi}}(\mathbf{x}) }_{L_fV(\xVec, \mathbf{\xi})} + \underbrace{L_GV_{\mathbf{\xi}}(\mathbf{x}) }_{L_GV(\xVec, \mathbf{\xi})}\mathbf{u} \leq -\gamma(V_0(\mathbf{x})) -\gamma'(V_{\mathbf{\xi}}(\mathbf{x})) + \delta \vspace{0.25em}\\
%         \quad \quad \hspace{1em}                 \underbrace{L_fh_0(\mathbf{x}) + L_Gh_0(\mathbf{x})\mathbf{\xi} - L_fV_{\mathbf{\xi}}(\mathbf{x}) }_{L_fh(\xVec, \mathbf{\xi})} + \underbrace{ - L_GV_{\mathbf{\xi}}(\mathbf{x}) }_{L_Gh(\xVec, \mathbf{\xi})}\mathbf{u} \geq -\alpha(h_0(\mathbf{x})) + \alpha'(V_{\mathbf{\xi}}(\mathbf{x}))
%     \end{array}
%  \label{eq:CLF-CBF-QP-relaxed-1Backstep}
% \end{equation}

% so to tkcche respective CLF and CBF is added a CLF contributing tho the copnvergence of \xi to k0

\subsection{Closed Form}
\label{sub:closed_form}

The objective of this thesis is to purpose a set of low computational effort techniques suitable to spacecrafts. Aiming a faster computation of the optimization problems solutions, any of the previous \glsxtrshort{QP}s (\ref{eq:CLF-CBF-QP}, \ref{eq:CLF-CBF-QP-relaxed}, \ref{eq:CLF-CBF-QP-relaxed-0Backstep}, \ref{eq:CLF-CBF-QP-relaxed-1Backstep}) have a closed-form\footnote{Explicar o que e closed form} solution which allows to calculate directly the respective solution instead of depending of a numerical solver. Given the \glsxtrshort{QP} optimization problem and a controller \(k(\xVec, \mathcolor{purple}{\mathbf{\xi}}): \mathbb{R}^n \mathcolor{purple}{\times \mathbb{R}^p} \to \mathbb{R}^m\) \textcolor{purple}{(if it is a second-order system)}:

\begin{equation}
    \begin{array}{l}
        k(\xVec \mathcolor{purple}{, \mathbf{\xi}}) = \underset{\uVec \in \mathbb{R}^m, \hspace{0.2em} \delta \in \mathbb{R}}{\mathrm{argmin}} \quad \hspace{-0.5em} \frac{1}{2} (\uVec^\top R \uVec + p\delta^2 ) \vspace{1em} \\ 
        \quad \hspace{0.5em}  st. \hspace{0.5em} \underbrace{\underbrace{L_fV(\xVec \mathcolor{purple}{, \mathbf{\xi}}) + \gamma(\mathcolor{orange}{V_0(\mathbf{x})}) \mathcolor{purple}{+ \gamma'(V_{\mathbf{\xi}}(\mathbf{x}, \mathbf{\xi}))}}_{F_V(\mathbf{x} \mathcolor{purple}{, \mathbf{\xi}})} + L_GV(\xVec \mathcolor{purple}{, \mathbf{\xi}})\mathbf{u} - \delta }_{g_1(\xVec \mathcolor{purple}{, \mathbf{\xi}})}\leq  0 \vspace{0.25em}\\
        \quad \quad \hspace{1em}                 \underbrace{\underbrace{L_fh(\xVec \mathcolor{purple}{, \mathbf{\xi}}) + \alpha(\mathcolor{orange}{h_0(\mathbf{x})}) \mathcolor{purple}{- \alpha'(V_{\mathbf{\xi}}(\mathbf{x}, \mathbf{\xi}))}}_{F_h(\mathbf{x} \mathcolor{purple}{, \mathbf{\xi}})} + L_Gh(\xVec \mathcolor{purple}{, \mathbf{\xi}})\mathbf{u}}_{-g_2(\xVec \mathcolor{purple}{, \mathbf{\xi}})} \geq 0 
    \end{array}
 \label{eq:CLF-CBF-QP-relaxed-generic}
\end{equation}

With a \glsxtrshort{CLF} \(V\), \glsxtrshort{CBF} \(h\), \(\mathbb{x}\in\mathbb{R}^n\), \textcolor{orange}{the first-order system \glsxtrshort{CLF}} \(\mathcolor{orange}{V_0} ( = V\) like in \ref{eq:CLF-CBF-QP}, \ref{eq:CLF-CBF-QP-relaxed} and \ref{eq:CLF-CBF-QP-relaxed-0Backstep}) \textcolor{purple}{, and if it is a second-order system with} \(\mathcolor{purple}{\mathbf{\xi}\in\mathbb{R}^p}\) \textcolor{purple}{(equivalent to the \glsxtrshort{QP} seen while doing Backstepping~\ref{eq:CLF-CBF-QP-relaxed-1Backstep})}. \\


Assuming equal control input wheights relative to each other (\(R = \mathbf{I}_{m \times m}\)) and just one \glsxtrshort{CLF} and \glsxtrshort{CBF} each, according to the \glsxtrfull{KKT} conditions\footnote{\href{https://en.wikipedia.org/wiki/Karush-Kuhn-Tucker_conditions}{KKT Conditions Wikipedia \cite{} (Click here to be Redirected)}} (to see the full deduction \ref{}):

\begin{equation}
    \mathbf{k}(\xVec \mathcolor{purple}{, \mathbf{\xi}}) =
    \begin{cases}
        \mathbf{k}_1(\xVec \mathcolor{purple}{, \mathbf{\xi}}), \quad (\xVec \mathcolor{purple}{, \mathbf{\xi}}) \in \mathcal{S}_1 \\
        \mathbf{k}_2(\xVec \mathcolor{purple}{, \mathbf{\xi}}), \quad (\xVec \mathcolor{purple}{, \mathbf{\xi}}) \in \mathcal{S}_2 \\
        \mathbf{k}_3(\xVec \mathcolor{purple}{, \mathbf{\xi}}), \quad (\xVec \mathcolor{purple}{, \mathbf{\xi}}) \in \mathcal{S}_3 \\
        \mathbf{0}_{m \times 1 } \quad  \hspace{0.25em}, \quad (\xVec \mathcolor{purple}{, \mathbf{\xi}}) \in \mathcal{S}_4
    \end{cases}
    \label{eq:closed-form_controller}
\end{equation}


Where each controller \(\mathbf{k}_i\), for \( i = 1, ..., 4 \), corresponds to the closed-form equation depending which constraints (\glsxtrshort{CLF} or \glsxtrshort{CBF}) are active what can be inferred by \(\mathcal{S}_i\), for \( i = 1, ..., 4 \). Via \glsxtrshort{KKT} conditions the respectives controllers and domains are obtained due to the \glsxtrshort{KKT} multipliers.\\

Auxiliary Variables:

\begin{align*}
    & L \sOrderArgB = L_GV \sOrderArgB \hspace{0.2em} L_Gh\sOrderArgB^{\top} \\
    & \Delta = L \sOrderArgB^2 (p^{-1} + ||L_Gh\sOrderArgB||^2)  \hspace{0.2em} || L_Gh\sOrderArgB||^2
\end{align*}

\glsxtrshort{KKT} multipliers:

\begin{align}
    & \lambda_1(\xVec \mathcolor{purple}{, \mathbf{\xi}}) = \Delta^{-1} (L \sOrderArgB \hspace{0.2em} F_h \sOrderArgB \hspace{0.2em} - ||L_Gh\sOrderArgB||^2  \hspace{0.2em}F_V \sOrderArgB)
    \notag\\
    & \lambda_2(\xVec \mathcolor{purple}{, \mathbf{\xi}}) = \Delta^{-1} (L \sOrderArgB F_V \sOrderArgB \hspace{0.2em} - (p^{-1} + ||L_Gh\sOrderArgB||^2)  \hspace{0.2em}F_h \sOrderArgB) 
    \label{eq:KKT-Multipliers_formulas}
\end{align}

Controller \(\mathbf{k}_i\) closed-form equation:

\begin{equation}
    \begin{array}{l}
    \mathbf{k}_1(\xVec \mathcolor{purple}{, \mathbf{\xi}}) = -(p^{-1} + ||L_GV \sOrderArgB||^2)^{-1} \hspace{0.2em} F_V\sOrderArgB \hspace{0.2em} L_GV\sOrderArgB^{\top} \vspace{0.5em} \\ 
    \mathbf{k}_2(\xVec \mathcolor{purple}{, \mathbf{\xi}}) = -||L_Gh \sOrderArgB||^{-2} \hspace{0.2em} F_h \sOrderArgB \hspace{0.2em} L_Gh \sOrderArgB^{\top} \vspace{0.5em} \\ 
    \mathbf{k}_3(\xVec \mathcolor{purple}{, \mathbf{\xi}}) = -\lambda_1\sOrderArgB L_GV \sOrderArgB^{\top} + \lambda_2\sOrderArgB L_Gh \sOrderArgB^{\top} \vspace{0.5em}
    \label{eq:closed-form_controller_formulas}
    \end{array} 
\end{equation}



Constraint Activation Subdomains \(\mathcal{S}_i\) ( based on the dual feasibilitie condition \href{https://en.wikipedia.org/wiki/Karush-Kuhn-Tucker_conditions}{\cite{}} ):

\begin{align}
    \intertext{If CLF is active }
    \mathcal{S}_1 \sOrderArgB = \{ \sOrderArgB \in \mathbb{R}^n \mathcolor{purple}{\times \mathbb{R}^p}: \lambda_1 \geq 0 \cap \lambda_2 < 0 \}
    \notag \\
    \intertext{If CBF is active}
    \mathcal{S}_2 \sOrderArgB = \{ \sOrderArgB \in \mathbb{R}^n \mathcolor{purple}{\times \mathbb{R}^p}: \lambda_1 < 0 \cap \lambda_2 \geq 0 \}
    \notag \\
    \label{eq:Constraints_Activation_Subdomains}
    \intertext{If CLF and CBF is active}
    \mathcal{S}_3 \sOrderArgB = \{ \sOrderArgB \in \mathbb{R}^n \mathcolor{purple}{\times \mathbb{R}^p}: \lambda_1 \geq 0 \cap \lambda_2 \geq 0 \}
    \notag\\
    \intertext{If Any is active}
    \mathcal{S}_4\sOrderArgB = \{ \sOrderArgB \in \mathbb{R}^n \mathcolor{purple}{\times \mathbb{R}^p}: \lambda_1 < 0 \cap \lambda_2 < 0 \}
    \notag
\end{align}


\subsection{Unifying \glsxtrshort{CBF}}
\label{sub:unifying_CBF}














\endinput




% \begin{center}
%   \fbox{\LARGE
%     This manual is outdated and must be revised!}
% \end{center}



% \begin{flushleft}
% \hspace*{0.5cm}“\verb!n015002t.ttf!”, “\verb!n015003t.ttf!”, and “\verb!n015006t.ttf!”
% \end{flushleft}
   

% \ref{it:project_available} above in Section~\ref{sub:with_a_local_latex_installation} (\nameref{sub:with_a_local_latex_installation}).


% subsection with_a_remote_cloud_based_service (end)


% \newcommand{\accessAllowed}{\includegraphics[align=c,width=1.9em]{access_allowed}}
% \newcommand{\accessForbiden}{\includegraphics[align=c,width=1.9em]{dont_touch}}
% \newcommand{\File}{\includegraphics[align=c,width=1.9em]{file}}
% \newcommand{\Folder}{\includegraphics[align=c,width=1.9em]{folder}}


% \bgroup
%     \rowcolors{1}{}{GhostWhite}
%       \begin{xltabular}{\textwidth}{>{\ttfamily}l>{\itshape}lcX}
%         \caption{The folders and files (top level).}
%         \label{tab:folders_and_files}\\
%         \toprule
%         \rowcolor{Gainsboro}%
%         Name & Type & Access & Contents \\
%         \midrule
% template.tex      & \File    & \accessForbiden &
% The main template file. You need to \emph{compile} this file with one of \pdfLaTeX, \XeLaTeX, or \LuaLaTeX\ to obtain the PDF file (”\texttt{template.pdf}”).  I recommend the usage of the ”\texttt{latexmk}” command or, if you use a UN*X-like OS, you may use ”\texttt{make}” (and the ggiven ”\texttt{Makefile}”).
% \\
% Config          & \Folder  & \accessAllowed &
% Configuration files.  Please customize your template by changing the files in this folder!
% \\
%         \bottomrule
%         \end{xltabular}
%     % \end{longtblr}
% \egroup


% \newcommand{\classoption}[4]{\textbf{#1=OPT}\newline\emph{\small#2}&\textbf{#3}\newline{\small#4}\\}
% \newcommand{\defaultopt}[1]{\mbox{$\Rightarrow$~\emph{Default: \texttt{#1}}}\newline}
% \newcommand{\defaultit}[1][default]{($\Leftarrow$~\emph{#1})}


% \bgroup
% \begin{xltabular}{\linewidth}{>{\hsize=.4\hsize\raggedright\arraybackslash}X>{\hsize=.6\hsize}X}
%   \toprule
% %----------------------------------------------------------------------
%   \classoption{doctype}%
%     {phd, phdprop, phdplan, msc, mscplan, bsc, plain}%
%     {The type of the document.}%
% 	{%
%     \begin{tabular}{@{}r@{ $\rightarrow$ }l@{}}
%         phd & PhD thesis \defaultit.\\
%     phdprop & PhD thesis proposal (for FCT-NOVA).\\
%     phdplan & PhD thesis plan.\\
%         msc & MSc thesis.\\
%     mscplan & MSc thesis plan.\\
%         bsc & BSc report.\\
%       plain & Other report.\\
%     \end{tabular}
%     }
% %----------------------------------------------------------------------
%     \midrule
%   \classoption{school}%
%   	{nova/fct
% 	}%
%     {Selection of the university and of the school (and degree variant).}%
%     {\defaultopt{school=nova/fct} }
% %----------------------------------------------------------------------
%     \midrule
%   \classoption{docstatus}%
%     {draft, provisional, final}%
%     {The current status of the document.}%
% 	{}
% %----------------------------------------------------------------------
%     \bottomrule
% \end{xltabular}
% \egroup


% \printbibliography[heading=subbibliography, segment=\therefsegment, title={\bibname\ for chapter~\thechapter}]

% \section{\glsfmtshort{novathesisclass}\ Class Options}
% \label{sec:package_options}

 % \todo[inline]{A a note in a line by itself.}







