%!TEX root = ../template.tex
%%%%%%%%%%%%%%%%%%%%%%%%%%%%%%%%%%%%%%%%%%%%%%%%%%%%%%%%%%%%%%%%%%%%
%% chapter2.tex
%% NOVA thesis document file
%%
%% Chapter with the template manual
%%%%%%%%%%%%%%%%%%%%%%%%%%%%%%%%%%%%%%%%%%%%%%%%%%%%%%%%%%%%%%%%%%%%

\typeout{NT FILE chapter2.tex}%


\chapter{Methodology}
\label{cha:methodology}

\glsresetall %Para em vez de mostrar o acrónimo voltar a mostrar o nome completo para relembrar



%Aqui falar do modelo generico e de algumas coisas que serviram de base para os proximos topicos e que não valem a pena uma secção, alem de introduzir o que ai vem

\section{\glsxtrshort{CLF-CBF}}
\label{sec:clf_cbf}

Considering a nonlinear control-affine system:

\begin{equation}
    \dot{\xVec} = f(\xVec) + g(\xVec)\uVec
    \label{eq:Nonlinear-First-Order-System} 
\end{equation}

with stet \(\xVec \in \mathbb{R}^n\), control input \(\uVec \in \mathbb{R}^m\), and locally Lipschitz continuos\footnote{Given two metric spaces (\(X, d_X\)) and (\(Y, d_Y\)), where \(d_X\) and \(d_Y\)  represent the \(X\) and \(Y\) metrics respectively, a function \(f:X \to Y\) is Lipschitz continuos if there exist a constant \(L \geq 0\) such that \(\forall (x_1,x_2) \in X\), \hspace{0.4em} \(d_Y(f(x_1),f(x_2))\leq L \hspace{0.2em} d_X(x_1,x_2)\)} functions \(f: \mathbb{R}^n \to \mathbb{R}^n\) and \(g: \mathbb{R}^n \to \mathbb{R}^{n \times m}\). Given a locally Lipschitz continuos \(k: \mathbb{R}^n \to \mathbb{R}^m\), it results in the closed-loop system:

\begin{equation}
    \dot{\xVec} = f(\xVec) + g(\xVec)k(\xVec)
    \label{eq:Nonlinear-First-Order-Closed-Loop-System} 
\end{equation}

For any initial condition \(\xVec_0 \in \mathbb{R}^n\), there is a maximal time interval \(\mathcal{I}(\xVec_0) = [0; t_{max}(\xVec_0)[ \) and unique continuosly differentiable solution \(\varphi: \mathcal{I}(\xVec_0) \to \mathbb{R}^n \) such that:

\begin{equation}
    \begin{array}{l}
        \dot{\varphi}(t) = f(\varphi(t))+ g(\varphi(t))k(\varphi(t)) \quad \forall t \in \mathcal{I}(\xVec_0)  \\
        \varphi(0) = \xVec_0
    \end{array}
 \label{eq:Nonlinear-First-Order-Closed-Loop-System-UniqueSol}
\end{equation}


\subsection{First-Order Formulation}
\label{sub:formulation}

\subsubsection{\glsxtrfull{CLF}}
\label{subsub:control_lyapunov_function}

Having the objective of global assymptotically stability of the system~\ref{eq:Nonlinear-First-Order-Closed-Loop-System}, is desired for the system converge to an equilibrium point \( \bar{\mathbf{x}} \). A solution comes by designing a positive definite function \( V: \mathbb{R}^n \rightarrow \mathbb{R}_{\geq 0} \) which converges to zero while the system converges to the respective equilibrium point, and for that, designing a control law that acts in order to achieve specifically the convergence of the function. Consequentially is pretended to the function \( V\mathbf(\mathbf{x}) \) to be continuos and differentiable:

\[\dot{V} (\mathbf{x}, \mathbf{u}) = \frac{\partial V}{\partial x}\dot{x} = L_fV(\mathbf{x}) + L_GV(\mathbf{x})\mathbf{u} \]

where \(L_fV\) and \(L_GV\) are the Lie derivatives of \(V\) and \(\mathbf{u}\) the control input vector resposible to manipulate the function, more precisely, its convergence rate.\par

That beeing said, a continuosly differentiable, positive definite function \( V: \mathbb{R}^n \rightarrow \mathbb{R}_{\geq 0}  \) wich \( \lim_{\mathbf{x} \to \bar{\mathbf{x}}}{V(\mathbf{x})} = 0 \) is a \glsxtrshort{CLF} if there exists a \( \gamma \in \mathcal{K}^e_{\infty}  \)\footnote{If a continuos functions \(\alpha:[-b,a] \to \mathbb{R}\) with \((a,b) > 0\), strictly increasing and \(\alpha(0) = 0\) then \(\alpha \in \mathcal{K}^e\) and if it adds \(\lim_{r \to \infty} \alpha(r) = \infty\) then \(\alpha \in \mathcal{K}^e_{\infty}\)}  such that for all \( \mathbf{x} \in \mathbb{R}^n \setminus \{\bar{\mathbf{x}}\} \):

\begin{equation}
 \inf_{\mathbf{u} \in \mathbb{R}^m} [L_fV(\mathbf{x}) + L_GV(\mathbf{x})\mathbf{u}] \leq -\gamma(V(\mathbf{x}))
 \label{eq:CLF}
\end{equation}

From \glsxtrshort{CLF} \(V\) and an adequate tuning of \( \gamma \) parameter results a control law followed by the controller \(K_{CLF}\):

\begin{equation}
 K_{CLF}(\mathbf{x}) = \{ \mathbf{u} \in \mathbb{R}^m: L_fV(\mathbf{x}) + L_GV(\mathbf{x})\mathbf{u} \leq -\gamma(V(\mathbf{x})) \}
 \label{eq:K-CLF}
\end{equation}

The stets generated by \(K_{CLF}(\mathbf{x}) \hspace{0.5em} \forall x \in \mathbb{R}^n \) refer to the possible control input values necessary to the \glsxtrshort{CLF} \(V(\mathbf{x})\) continuosly converge to zero while the closed-loop system~\ref{eq:Nonlinear-First-Order-Closed-Loop-System} to the equilibrium point, but more slowly as is close to reach the objective. \\


\subsubsection{\glsxtrfull{CBF}}
\label{subsub:control_barrier_function}

Now having the objective of garantee safety to system~\ref{eq:Nonlinear-First-Order-Closed-Loop-System}, is aimed to keep the state inside a safe set \(\mathcal{C}\) (foward invariant\footnote{A set \(\mathcal{S}\) is said foward invariant if the initial system state \( x_0 \in \mathcal{S}\) and so \(x(t) \in \mathcal{S} \hspace{0.25em} \forall t \geq 0\) }) defined accordingly the problem needs. The notion of safety, i.e., the safe set \(\mathcal{C}\) is defined as the 0-superlevel\footnote{For function \(f: \mathbb{R}^n \to \mathbb{R}\), its superlevel set is given by \(\mathcal{S}(f) = \{(x_1,...,x_n): f(x_1,...,x_n) \geq c\}\), where having \(c = 0\) corresponds to the 0-superlevel set of \(f\)} set of a continuosly differentiable function \( h: \mathbb{R}^n \rightarrow \mathbb{R}  \):

\begin{equation}
    \begin{array}{l}
        \qquad   \mathcal{C} = \{ \mathbf{x} \in \mathbb{R}^n : h( \mathbf{x} )\geq 0\} \\
        \qquad \hspace{-0.5em}  \partial\mathcal{C} = \{ \mathbf{x} \in \mathbb{R}^n : h( \mathbf{x} ) = 0\} \\
        int(\mathcal{C}) = \{ \mathbf{x} \in \mathbb{R}^n : h( \mathbf{x} ) > 0\}
    \end{array}
 \label{eq:safe-set}
\end{equation}


In order to mantain \(h( \mathbf{x} )\) positive, there is a need to have a control law and but first to be able to interface with the function. Since \(h( \mathbf{x} )\) is differentiable:  

\[\dot{h}(\mathbf{x}, \mathbf{u}) = \frac{\partial h}{\partial x}\dot{x} = L_fh(\mathbf{x}) + L_Gh(\mathbf{x})\mathbf{u} \]

where \(L_fh\) and \(L_Gh\) are the Lie derivatives of \(h\) and \(\mathbf{u}\) the control input vector resposible to manipulate the function, more precisely, its time derivative.\par

So, a differentiable function \( h: \mathbb{R}^n \rightarrow \mathbb{R}  \) is a \glsxtrshort{CBF} if there exists a \( \alpha \in \mathcal{K}^e_{\infty}  \) such that for all \( \mathbf{x} \in \mathbb{R}^n \setminus \{\bar{\mathbf{x}}\} \):

\begin{equation}
 \sup_{\mathbf{u} \in \mathbb{R}^m} [L_fh(\mathbf{x}) + L_Gh(\mathbf{x})\mathbf{u}] \geq -\alpha(h( \mathbf{x} ))
 \label{eq:CBF}
\end{equation}

Similar to the \glsxtrshort{CLF}, from \glsxtrshort{CBF} \(h\) and an adequate tuning of \( \alpha \) parameter results a control law encapsulated by the controller \(K_{CBF}\):

\begin{equation}
 K_{CBF}(\mathbf{x}) = \{ \mathbf{u} \in \mathbb{R}^m: L_fh(\mathbf{x}) + L_Gh(\mathbf{x})\mathbf{u} \geq -\alpha(h( \mathbf{x} )) \}
 \label{eq:K-CBF}
\end{equation}

\newpage %Só para ajeitar

The controller \(K_{CLF}(\mathbf{x}) \hspace{0.5em} \forall x \in \mathbb{R}^n \) renders control input sets that keep the closed-loop system~\ref{eq:Nonlinear-First-Order-Closed-Loop-System} safe with respect to the set \(\mathcal{C}\). The controller by defenition is able to evolve the closed-loop system more freely when \(\dot{h}( \mathbf{x} ) > 0\) and more cautiously as \(h( \mathbf{x} ) \) shortens and \(\dot{h}( \mathbf{x} ) < 0\), if \(h( \mathbf{x} ) < 0\) it means is in the unsafe region, so \(\dot{h}( \mathbf{x}) > 0\) implies converging continuosly to the safe set \(\mathcal{C}\). \\ 




\subsubsection{Quadratic Program Formulation}
\label{subsub:quadratic_program_formulation}

Now having the objective of not only achieve safety but also assymptotically stability through the usage of \glsxtrshort{CLF} and \glsxtrshort{CBF} it renders a set of control inputs able to achieve the respective purposes.  

Using the conditions imposed by \glsxtrshort{CLF} and \glsxtrshort{CBF} as constraints, is presented an optimization problem  via \glsxtrfull{QP}~\footnote{\glsxtrlong{QP} consists of solving a mathematical optimization problem with a quadratic objective function}, given a locally Lipschitz continuos controller \(k: \mathbb{R}^n \rightarrow \mathbb{R}^m \):

\begin{equation}
    \begin{array}{l}
        k(\xVec) = \underset{\uVec \in \mathbb{R}^m }{\mathrm{argmin}} \quad \hspace{-0.5em} \frac{1}{2} (\uVec^\top R \uVec) \vspace{1em} \\  
        \quad \hspace{0.5em}  st. \hspace{0.5em} L_fV(\mathbf{x}) + L_GV(\mathbf{x})\mathbf{u} \leq -\gamma(V(\mathbf{x})) \vspace{0.25em} \\ 
        \quad \quad \hspace{1em} L_fh(\mathbf{x}) + L_Gh(\mathbf{x})\mathbf{u} \geq -\alpha(h(\mathbf{x}))
    \end{array}
 \label{eq:CLF-CBF-QP}
\end{equation}

The sets of control inputs restricted to the optimization problem are given by \(K_{CLF-CBF}(\xVec) = K_{CLF}(\xVec) \cap  K_{CBF}(\xVec)\) hopefully non-empty which can happen if the conditions are in conflict. \par

In case of conditions' conflict, in order to the closed loop system~\ref{eq:Nonlinear-First-Order-Closed-Loop-System} controller have a solution from the \glsxtrshort{QP}, since safety is critical, using a slack variable \(\delta\)  allows to relax the stability condition imposed by the \glsxtrshort{CLF} and highly contribute to obtaining a solution while keeping system safe:

\begin{equation}
    \begin{array}{l}
        k(\xVec) = \underset{\uVec \in \mathbb{R}^m, \hspace{0.2em} \delta \in \mathbb{R}}{\mathrm{argmin}} \quad \hspace{-0.5em} \frac{1}{2} (\uVec^\top R \uVec + p\delta^2 ) \vspace{1em} \\ 
        \quad \hspace{0.5em}  st. \hspace{0.5em} L_fV(\mathbf{x}) + L_GV(\mathbf{x})\mathbf{u} \leq -\gamma(V(\mathbf{x})) + \delta \vspace{0.25em}\\
        \quad \quad \hspace{1em} L_fh(\mathbf{x}) + L_Gh(\mathbf{x})\mathbf{u} \geq -\alpha(h(\mathbf{x}))
    \end{array}
 \label{eq:CLF-CBF-QP-relaxed}
\end{equation}

Having \(\delta\) as an optimization variable not only allows to balance between the input values exceed usage and the level of relaxation but also to not have a fixed \(\delta\) value which would result for all states not so faster convergence and to the neighborhood of the equilibrium point instead of the actual target point. \\

%tikz daquele esquema
\begin{tcolorbox}[colback=blue!5!white,colframe=blue!35!white,title=Notes:]
\begin{itemize}
    \item If it's not pretended to enforce stability or safety, just remove from the optimization problem the respective constraint responsible to it
\end{itemize}
\end{tcolorbox}

\newpage %Só para ajeitar


\subsection{Backstepping (Second-Order Formulation)}
\label{sub:backstepping}

\subsubsection{Second-Order Systems}
\label{subsub:higher_order_systems}

Usually control systems are governed by higher-order systems, such as some second-order cyber-physical systems controlling the position by applying some force to manipulate the speed that is felt in the position just instants after.  Higher order systems' outputs variation is felt by an multiple integrative effect, by an integration of each state affecting the next state dynamic until reach the so wanted output. Considering a non-linear system in strict-feedback form:

\begin{subequations}
    
    \label{eq:HH-NL-System}
    \begin{align}
        \dot{\xVec} = f_0(\xVec) + g_0(\xVec) \mathbf{\xi} 
        \label{eq:HH-NL-System-1stOrder}\\
        \dot{ \mathbf{\xi} } =  f_1(\xVec, \mathbf{\xi}) + g_1(\xVec, \mathbf{\xi}) \uVec
        \label{eq:HH-NL-System-2ndOrder}
    \end{align}
\end{subequations}

with \(\mathbf{x} \in \mathbb{R}^n \), \(\mathbf{\xi} \in \mathbb{R}^p \), and \(\mathbf{u} \in \mathbb{R}^m \), and functions \(f_0: \mathbb{R}^n \to \mathbb{R}^n \), \(g_0: \mathbb{R}^n \to \mathbb{R}^{n \times p} \), \(f_1: \mathbb{R}^n \times \mathbb{R}^p \to \mathbb{R}^p \) and \(g_1: \mathbb{R}^n \times \mathbb{R}^p \to \mathbb{R}^{ p \times m} \) locally Lipschitz continuos on their respective domains. \par
Given a locally Lipschitz continuos feedback controller \(\mathbf{k}: \mathbb{R}^n \times \mathbb{R}^p \to \mathbb{R}^m  \) the closed-loop system is:

\begin{equation}
    \begin{array}{l}
        \dot{\xVec} = f_0(\xVec) + g_0(\xVec) \mathbf{\xi}\\ 
        \dot{ \mathbf{\xi} } =  f_1(\xVec, \mathbf{\xi}) + g_1(\xVec, \mathbf{\xi}) \mathbf{k}(\xVec, \mathbf{\xi})
    \end{array}
 \label{eq:HH-NL-CL-System}
\end{equation}

for any initial states \((\xVec_0, \mathbf{\xi}_0) \in \mathbb{R}^n \times \mathbb{R}^p\) exists a maximum time interval \(I(\xVec_0, \mathbf{\xi}_0) \subseteq \mathbb{R}_{\geq 0}  \) and a unique solution \( \varphi = (\varphi_{\mathbf{x}}, \varphi_{\mathbf{\xi}}) \) satisfying \ref{eq:Nonlinear-First-Order-Closed-Loop-System-UniqueSol} \(\forall t \in I(\xVec_0, \mathbf{\xi}_0) \). \\


\subsubsection{Purposal  Solution Obtaining}
\label{subsub:quadratic_program_formulation}

Since having a second order system~\ref{eq:HH-NL-System} the access to \(\mathbf{\xi}\) is not direct, we must "backstepp" in order to be able to control \(\uVec\), affect \(\mathbf{\xi}\) and consequentially control the wanted dynamic \(\dot{\xVec}\). The strategy passes by obtaining the solution if it was a first order system~\ref{eq:Nonlinear-First-Order-System} and if \(\mathbf{\xi}\) could be directly controlled (i.e. as part of the first order system control input), so given a differentiable and locally Lipschitz continuos \(k_0(\xVec):\mathbb{R}^n \to \mathbb{R}^p\) (that encompasses the "controllable \(\mathbf{\xi}\)" ), the closed-loop system:

\begin{equation}
    \dot{\xVec} = f_0(\xVec) + g_0(\xVec) k_0(\xVec)\\ 
 \label{eq:NL-CL-System-0Backstep}
\end{equation}

That beeing said, the first order solution could be obtained via \glsxtrshort{QP}, so supposing a \glsxtrshort{CLF} \(V_0\),  \glsxtrshort{CBF} \(h_0\),  a differentiable and locally Lipschitz continuos described just now \(k_0(\xVec)\) and \((\gamma_0, \alpha_0) \in \mathcal{K}^e_\infty \times \mathcal{K}^e_\infty\):

\begin{equation}
    \begin{array}{l}
        k_0(\xVec) = \underset{\uVec \in \mathbb{R}^p, \hspace{0.2em} \delta \in \mathbb{R}}{\mathrm{argmin}} \quad \hspace{-0.5em} \frac{1}{2} (\uVec^\top R_0 \uVec + p_0\delta^2 ) \vspace{1em} \\ 
        \quad \hspace{0.5em}  st. \hspace{0.5em} L_fV_0(\mathbf{x}) + L_GV_0(\mathbf{x})\mathbf{u} \leq -\gamma_0(V_0(\mathbf{x})) + \delta \vspace{0.25em}\\
        \quad \quad \hspace{1em} L_fh_0(\mathbf{x}) + L_Gh_0(\mathbf{x})\mathbf{u} \geq -\alpha_0(h_0(\mathbf{x}))
    \end{array}
 \label{eq:CLF-CBF-QP-relaxed-0Backstep}
\end{equation}


The solution is equivalent to a value of \(\mathbf{\xi}\) closest to an optimal value taken by it that would impose safety and contribute to stability not only to the first-order system~\ref{eq:NL-CL-System-0Backstep} but also to the actual second-order system. Although like it was said is not possible to have direct access to \(\mathbf{\xi}\), the backstepping technique is essentially the second-order trying to converge to the first order solution. Supposing there is \( V_{\mathbf{\xi}}(\xVec, \mathbf{\xi}): \mathbb{R}^n \times \mathbb{R}^p \rightarrow \mathbb{R}_{\geq 0} \), the new \glsxtrshort{CLF} \(V(\xVec, \mathbf{\xi})\) and \glsxtrshort{CBF} \(h(\xVec, \mathbf{\xi})\) are:
\begin{subequations}
    \label{eq:Backstepp_CF}
    \begin{align}
        V_{\mathbf{\xi}}(\xVec, \mathbf{\xi}) = \frac{1}{2}|| \mathbf{\xi} -  k_0(\xVec) ||^2 
        \label{eq:Backstepp_prejudice}\\
        V(\xVec, \mathbf{\xi}) = V_0(\xVec) + V_{\mathbf{\xi}}(\xVec, \mathbf{\xi}) 
        \label{eq:Backstepp_V}\\
        h(\xVec, \mathbf{\xi}) = h_0(\xVec) - V_{\mathbf{\xi}}(\xVec, \mathbf{\xi})
        \label{eq:Backstepp_h}
    \end{align}
\end{subequations}


Finally, via \Glsxtrshort{QP}, given a differentiable and locally Lipschitz continuos \(k(\xVec, \mathbf{\xi}):\mathbb{R}^n \times \mathbb{R}^p \to \mathbb{R}^m\):

\begin{equation}
    \begin{array}{l}
        k(\xVec, \mathbf{\xi}) = \underset{\uVec \in \mathbb{R}^p, \hspace{0.2em} \delta \in \mathbb{R}}{\mathrm{argmin}} \quad \hspace{-0.5em} \frac{1}{2} (\uVec^\top R \uVec + p\delta^2 ) \vspace{1em} \\ 
        \quad \hspace{0.5em}  st. \hspace{0.5em} L_fV(\xVec, \mathbf{\xi}) + L_GV(\xVec, \mathbf{\xi})\mathbf{u} \leq -\gamma(V_0(\mathbf{x})) -\gamma'(V_{\mathbf{\xi}}(\mathbf{x}, \mathbf{\xi})) + \delta \vspace{0.25em}\\
        \quad \quad \hspace{1em}                 L_fh(\xVec, \mathbf{\xi}) + L_Gh(\xVec, \mathbf{\xi})\mathbf{u} \geq -\alpha(h_0(\mathbf{x})) + \alpha'(V_{\mathbf{\xi}}(\mathbf{x}, \mathbf{\xi}))
    \end{array}
 \label{eq:CLF-CBF-QP-relaxed-1Backstep}
\end{equation}

It's expected that  \( \gamma << \gamma'\)  contributing to a faster convergence of \(\xi\) and more similar behaviour of the system~\ref{eq:HH-NL-CL-System} relative to the first-order system~\ref{eq:NL-CL-System-0Backstep} and mitigate the chances of the system~\ref{eq:HH-NL-CL-System} not be safe in relation to \(\mathcal{C}_0\). \\

\begin{tcolorbox}[colback=blue!5!white,colframe=blue!35!white,title=Notes:]
\begin{itemize}
    \item If it's not pretended to enforce stability or safety, just remove from both optimization problem (\ref{eq:CLF-CBF-QP-relaxed-0Backstep} and \ref{eq:CLF-CBF-QP-relaxed-1Backstep}) the respective constraint responsible to it
    \item If it's a higher-order (than second) system, then, assuming it's a third-order system, the solution obtained in \ref{eq:CLF-CBF-QP-relaxed-1Backstep} is the "new \(k_0\)" and the third-order solution tries to converge to the second-order one (while this one is converging to the first-order solution)
\end{itemize}
\end{tcolorbox} 



\subsection{Closed Form}
\label{sub:closed_form}

The objective of this thesis is to purpose a set of low computational effort techniques suitable to spacecrafts. Aiming a faster computation of the optimization problems solutions, any of the previous \glsxtrshort{QP}s (\ref{eq:CLF-CBF-QP}, \ref{eq:CLF-CBF-QP-relaxed}, \ref{eq:CLF-CBF-QP-relaxed-0Backstep}, \ref{eq:CLF-CBF-QP-relaxed-1Backstep}) have a closed-form\footnote{An closed-form expression is expressed by a limit number and well defined operations (elementary functions)} solution which allows to calculate directly the respective solution instead of depending of a numerical solver. Given the \glsxtrshort{QP} optimization problem and a controller \(k(\xVec, \mathcolor{gray}{\mathbf{\xi}}): \mathbb{R}^n \mathcolor{gray}{\times \mathbb{R}^p} \to \mathbb{R}^m\) \textcolor{gray}{(if it is a second-order system)}:

\begin{equation}
    \begin{array}{l}
        k(\xVec \mathcolor{gray}{, \mathbf{\xi}}) = \underset{\uVec \in \mathbb{R}^m, \hspace{0.2em} \delta \in \mathbb{R}}{\mathrm{argmin}} \quad \hspace{-0.5em} \frac{1}{2} (\uVec^\top R \uVec + p\delta^2 ) \vspace{1em} \\ 
        \quad \hspace{0.5em}  st. \hspace{0.5em} \underbrace{L_fV(\xVec \mathcolor{gray}{, \mathbf{\xi}}) + \gamma(\mathcolor{orange}{V_0(\mathbf{x})}) \mathcolor{gray}{+ \gamma'(V_{\mathbf{\xi}}(\mathbf{x}, \mathbf{\xi}))}}_{F_V(\mathbf{x} \mathcolor{gray}{, \mathbf{\xi}})} + L_GV(\xVec \mathcolor{gray}{, \mathbf{\xi}})\mathbf{u} - \delta \leq  0 \vspace{0.25em}\\
        \quad \quad \hspace{1em}                 \underbrace{L_fh(\xVec \mathcolor{gray}{, \mathbf{\xi}}) + \alpha(\mathcolor{orange}{h_0(\mathbf{x})}) \mathcolor{gray}{- \alpha'(V_{\mathbf{\xi}}(\mathbf{x}, \mathbf{\xi}))}}_{F_h(\mathbf{x} \mathcolor{gray}{, \mathbf{\xi}})} + L_Gh(\xVec \mathcolor{gray}{, \mathbf{\xi}})\mathbf{u} \geq 0 
    \end{array}
 \label{eq:CLF-CBF-QP-relaxed-generic}
\end{equation}

With a \glsxtrshort{CLF} \(V\), \glsxtrshort{CBF} \(h\), \(\mathbf{x}\in\mathbb{R}^n\), \textcolor{orange}{the first-order system \glsxtrshort{CLF}} \(\mathcolor{orange}{V_0} ( = V\) like in \ref{eq:CLF-CBF-QP}, \ref{eq:CLF-CBF-QP-relaxed} and \ref{eq:CLF-CBF-QP-relaxed-0Backstep}) \textcolor{gray}{, and if it is a second-order system with} \(\mathcolor{gray}{\mathbf{\xi}\in\mathbb{R}^p}\) \textcolor{gray}{(equivalent to the \glsxtrshort{QP} seen while doing Backstepping~\ref{eq:CLF-CBF-QP-relaxed-1Backstep})}. \\


Using \cite{matias2025hybrid} as a reference, assuming equal control input wheights relative to each other (\(R = \mathbf{I}_{m \times m}\)) and just one \glsxtrshort{CLF} and \glsxtrshort{CBF} each, according to the \glsxtrfull{KKT} conditions\footnote{\href{https://en.wikipedia.org/wiki/Karush-Kuhn-Tucker_conditions}{KKT Conditions Wikipedia (Click here to be Redirected)} \label{foot: KKT_Conditions}} (to see the full deduction \ref{app:CL_QP_CLF-CBF}):

\begin{equation}
    \mathbf{k}(\xVec \mathcolor{gray}{, \mathbf{\xi}}) =
    \begin{cases}
        \mathbf{k}_1(\xVec \mathcolor{gray}{, \mathbf{\xi}}), \quad (\xVec \mathcolor{gray}{, \mathbf{\xi}}) \in \mathcal{S}_1 \\
        \mathbf{k}_2(\xVec \mathcolor{gray}{, \mathbf{\xi}}), \quad (\xVec \mathcolor{gray}{, \mathbf{\xi}}) \in \mathcal{S}_2 \\
        \mathbf{k}_3(\xVec \mathcolor{gray}{, \mathbf{\xi}}), \quad (\xVec \mathcolor{gray}{, \mathbf{\xi}}) \in \mathcal{S}_3 \\
        \mathbf{0}_{m \times 1 } \quad  \hspace{0.25em}, \quad (\xVec \mathcolor{gray}{, \mathbf{\xi}}) \in \mathcal{S}_4
    \end{cases}
    \label{eq:closed-form_controller}
\end{equation}


Where each controller \(\mathbf{k}_i\), for \( i = 1, ..., 4 \), corresponds to the closed-form equation depending which constraints (\glsxtrshort{CLF} or \glsxtrshort{CBF}) are active what can be inferred by \(\mathcal{S}_i\), for \( i = 1, ..., 4 \). Via \glsxtrshort{KKT} conditions the respectives controllers and domains are obtained due to the \glsxtrshort{KKT} multipliers.\\

Auxiliary Variables:

\begin{align*}
    & L \sOrderArgB = L_GV \sOrderArgB \hspace{0.2em} L_Gh\sOrderArgB^{\top} \\
    & \Delta = L \sOrderArgB^2 (p^{-1} + ||L_Gh\sOrderArgB||^2)  \hspace{0.2em} || L_Gh\sOrderArgB||^2
\end{align*}

\glsxtrshort{KKT} multipliers:

\begin{align}
    & \lambda_1(\xVec \mathcolor{gray}{, \mathbf{\xi}}) = \Delta^{-1} (L \sOrderArgB \hspace{0.2em} F_h \sOrderArgB \hspace{0.2em} - ||L_Gh\sOrderArgB||^2  \hspace{0.2em}F_V \sOrderArgB)
    \notag\\
    & \lambda_2(\xVec \mathcolor{gray}{, \mathbf{\xi}}) = \Delta^{-1} (L \sOrderArgB F_V \sOrderArgB \hspace{0.2em} - (p^{-1} + ||L_Gh\sOrderArgB||^2)  \hspace{0.2em}F_h \sOrderArgB) 
    \label{eq:KKT-Multipliers_formulas}
\end{align}

Controller \(\mathbf{k}_i\) closed-form equation:

\begin{equation}
    \begin{array}{l}
    \mathbf{k}_1(\xVec \mathcolor{gray}{, \mathbf{\xi}}) = -(p^{-1} + ||L_GV \sOrderArgB||^2)^{-1} \hspace{0.2em} F_V\sOrderArgB \hspace{0.2em} L_GV\sOrderArgB^{\top} \vspace{0.5em} \\ 
    \mathbf{k}_2(\xVec \mathcolor{gray}{, \mathbf{\xi}}) = -||L_Gh \sOrderArgB||^{-2} \hspace{0.2em} F_h \sOrderArgB \hspace{0.2em} L_Gh \sOrderArgB^{\top} \vspace{0.5em} \\ 
    \mathbf{k}_3(\xVec \mathcolor{gray}{, \mathbf{\xi}}) = -\lambda_1\sOrderArgB L_GV \sOrderArgB^{\top} + \lambda_2\sOrderArgB L_Gh \sOrderArgB^{\top} \vspace{1.5em}
    \label{eq:closed-form_controller_formulas}
    \end{array} 
\end{equation}



Constraint Activation Subdomains \(\mathcal{S}_i\) ( based on the dual feasibilitie condition~\footref{foot: KKT_Conditions}):

\begin{align}
    \intertext{\hspace{4em} If CLF is active: }
    \mathcal{S}_1 \sOrderArgB = \{ \sOrderArgB \in \mathbb{R}^n \mathcolor{gray}{\times \mathbb{R}^p}: \lambda_1 \geq 0 \cap \lambda_2 < 0 \}
    \notag \\
    \intertext{\hspace{4em} If CBF is active:}
    \mathcal{S}_2 \sOrderArgB = \{ \sOrderArgB \in \mathbb{R}^n \mathcolor{gray}{\times \mathbb{R}^p}: \lambda_1 < 0 \cap \lambda_2 \geq 0 \}
    \notag \\
    \label{eq:Constraints_Activation_Subdomains}
    \intertext{\hspace{4em} If CLF and CBF is active:}
    \mathcal{S}_3 \sOrderArgB = \{ \sOrderArgB \in \mathbb{R}^n \mathcolor{gray}{\times \mathbb{R}^p}: \lambda_1 \geq 0 \cap \lambda_2 \geq 0 \}
    \notag\\
    \intertext{\hspace{4em} If Any is active:}
    \mathcal{S}_4\sOrderArgB = \{ \sOrderArgB \in \mathbb{R}^n \mathcolor{gray}{\times \mathbb{R}^p}: \lambda_1 < 0 \cap \lambda_2 < 0 \}
    \notag
\end{align}


\subsection{Unifying \glsxtrshort{CBF}}
\label{sub:unifying_CBF}

The previous controllers synthesized control inputs based on just one \glsxtrshort{CBF}, i.e., those controllers garantee safety w.r.t. a single safe set \(\mathcal{C}\), but given diffrent scenarios, it could exist more complex safety specifications and the need to impose more than one safety contraints simultaneously such as \glsxtrshort{CBF} \(h_i(\xVec)\), for \(i \in  \mathcal{I}=\{1,...N\}\):

\begin{align}
    & \mathcal{C}_i = \{ \xVec \in \mathbb{R}_n: h_i(\xVec) \geq 0\}
    \label{eq:Multiple_Safe_Sets} \\ 
    \notag\\
    & \mathcal{C} = \bigcap_{i \in \mathcal{I}} \mathcal{C}_i 
    \label{eq:Safe_Set_of_multiple}
\end{align}

But like it was said before regarding having more than one constraint could leave to conflict problems, empty safe sets (\(\mathcal{C} = \emptyset\)) and an inexistent set of acceptable control inputs (\(\mathbf{k}\sOrderArgB = \emptyset\)). Using \cite{molnar2023composing} as guidance, in order to avoid conflicts, merges all \glsxtrshort{CBF} into a single one that is able to capture all those safety notions and allow for even more complex safety specifications and to the resultant constraint fit the already obtained \glsxtrshort{CLF-CBF} closed-form solution~\ref{eq:closed-form_controller}. \\


\subsubsection{First-Order System}
\label{subsub:uniCBF_firstOrder_system}

The more complex safety specifications are defined via boolean logical operations between 0-superlevel sets of the diffrent \glsxtrshort{CBF}. Those operations are tipically seen in constrained optimization problems, like in \glsxtrshort{QP} of the form:


\begin{equation}
    \begin{array}{l}
        k(\xVec) = \underset{\uVec \in \mathbb{R}^p, \hspace{0.2em} \delta \in \mathbb{R}}{\mathrm{argmin}} \quad \hspace{-0.5em} \frac{1}{2} (\uVec^\top R \uVec + p\delta^2 ) \vspace{1em} \\ 
        \quad \hspace{0.5em}  st. \hspace{0.5em} L_fV(\xVec) + L_GV(\xVec)\mathbf{u} \leq -\gamma(V(\mathbf{x})) + \delta \vspace{0.25em}\\
        \quad \quad \hspace{1em}                 L_fh_i(\xVec) + L_Gh_i(\xVec)\mathbf{u} \geq -\alpha_i(h_i(\mathbf{x}))  \quad \forall i \in \mathcal{I}    
    \end{array}
 \label{eq:CLF-MultipleCBF-QP-relaxed-generic}
\end{equation}

Where the resultant safe set~\ref{eq:Safe_Set_of_multiple} of the system refers to the intersection of all safe sets~\ref{eq:Multiple_Safe_Sets}. More than be propitious to be not solvable, the formulation itself of the optimization problem~\ref{eq:CLF-MultipleCBF-QP-relaxed-generic} is limitated by no other logical operation than AND. \par

Among the the logical operations that could be applied to \glsxtrshort{CBF}s, there are:

\begin{description}

    \item[Identity ( Class-\(\mathcal{K}^e\)):] 
    \begin{subequations}
        \begin{align}
            \mathcal{C}_i &= \{(\xVec, \alpha_i) \in \mathbb{R}^n \times \mathcal{K}^e: \alpha_i(h_i(\xVec) \geq 0)\} = 
            \label{eq:Indentity_set} \\
                          &= \{\xVec \in \mathbb{R}^n: h_i(\xVec) \geq 0\}
            \label{eq:Gen_Safe_set} 
        \end{align}
        
    \end{subequations}

    \item[Complement Set (Negation):] 
    
    \begin{equation}
        \bar{\mathcal{C}_i} = \{\xVec \in \mathbb{R}^n: -h_i(\xVec) \geq 0\} 
        \label{eq:Negation_set} 
    \end{equation}

    \item[Intersection of Sets (AND):] 

        \begin{equation}
            \begin{array}{rl}      
            \underset{i \in \mathcal{I}}{\bigcap} \mathcal{C}_i &= \{\xVec \in \mathbb{R}^n: h_i(\xVec) \geq 0 \hspace{0.5em} \forall i \in \mathcal{I}\}  \\
                                                      &= \{\xVec \in \mathbb{R}^n: \underset{i \in \mathcal{I}}{\min}(h_i(\xVec)) \geq 0 \} 
            \end{array}
            \label{eq:Intersection_set}
        \end{equation}

    \item[Union of Sets (OR):] 
    
        \begin{equation}
            \begin{array}{rl}
                \underset{i \in \mathcal{I}}{\bigcup} \mathcal{C}_i &= \{\xVec \in \mathbb{R}^n: \exists i \in \mathcal{I} \hspace{0.5em} st. \hspace{0.5em} h_i(\xVec) \geq 0 \}  \\
                                                          &= \{\xVec \in \mathbb{R}^n: \underset{i \in \mathcal{I}}{\max}(h_i(\xVec)) \geq 0 \} 
            \end{array}
            \label{eq:Union_set}
        \end{equation}
\end{description}

The logical operations identity, complement, intersection and union are represented by respectively, class-\(\mathcal{K}^e\) functions, negation, min and max operations. Although the class-\(\mathcal{K}^e\) (\(\alpha_i(h_i(\xVec))\)) and negation (\(-h_i(\xVec)\)) functions which are continuosly differentiable and valid as \glsxtrshort{CBF}, contrary to the AND and OR functions, \(\min_{ i \in \mathcal{I}} (h_i(\xVec))\) and \(\max_{ i \in \mathcal{I}} (h_i(\xVec))\) that aren't continuosly differentiable and therefore not \glsxtrshort{CBF} candidates.  The \(\max\) and \(\min\) functions capacitate combining safe sets defined by multiple \glsxtrshort{CBF} (accordingly to the operation) and to obtain a new safety specification as is expected to see while merging multiple \glsxtrshort{CBF} into a single one. So, in order to viablize the AND and OR operation and achieve the merge of \glsxtrshort{CBF}s, the \(min\) and \(max\) functions can be replaced via smooth-aproximation, keeping their properties with just a relative small degradation of accuracy, by continuosly differentiable \textbf{log-sum-exp functions}:


\begin{description}

    \item[Intersection of Sets \(\setminus\)AND \(\setminus\) Minimum]

    \begin{align}
        h(\xVec) &= \frac{-1}{k}ln(\sum_{i \in \mathcal{I}} e^{-k \hspace{0.2em} h_i(\xVec)}) 
        \label{eq:Min_smoothFunc} \\
                 &\approx \underset{i \in \mathcal{I}}{\min} (h_i(\xVec))
        \notag  
    \end{align}

    Beeing \(k \in \mathbb{R}_{>0}\) as smoothing parameter (if \(k<0\) then is equivalent to the smooth-aproximation of \(\max_{i \in \mathcal{I}}(h_i(\xVec))\) ). 

    \begin{subequations}
        \begin{align}
            &\dot{h}(\xVec , \uVec) = \underbrace{\sum_{i \in \mathcal{I}} [\lambda_i \hspace{0.2em} L_fh_i(\xVec)]}_{Lfh(\xVec)} + (\underbrace{\sum_{i \in \mathcal{I}} [\lambda_i \hspace{0.2em} L_Gh_i(\xVec)]}_{LGh(\xVec)})\uVec 
            \label{eq:Min_smooth_derivative} \\ \notag \\
            &\lambda_i(\xVec) = e^{-k(h_i(\xVec)-h(\xVec))}
            \label{eq:Min_lambda_smoothFunc}
        \end{align}
    \end{subequations}

    With \(\sum_{i \in \mathcal{I}} \lambda_i(\xVec) = 1\) \par
    Encountering an under-aproximation of \(h\) relative to the \(\min\) function, given the safe set \(\mathcal{C}\) defined based on \ref{eq:safe-set} it verifies:

    \begin{subequations}
        \begin{align}
            &\underset{i \in \mathcal{I}}{\min} (h_i(\xVec)) \geq h(\xVec) \quad \forall \xVec \in \mathbb{R}^n 
            \label{eq:Intersection_CBF_Comparation} \\
            &\mathcal{C} \subseteq \bigcap_{i \in \mathcal{I}} \mathcal{C}_i
            \label{eq:Intersection_SafeSet_Comparation}
        \end{align}
    \end{subequations}

    But, \(\hspace{0.4em} \lim_{k \to \infty} h(\xVec) = {\min}_{i \in \mathcal{I}} (h_i(\xVec)) \hspace{0.4em}\) and \( \hspace{0.4em} \lim_{k \to \infty} \mathcal{C} = \bigcap_{i \in \mathcal{I}} \mathcal{C}_i\) \par

    On par with that, the new \glsxtrshort{CBF} \(h(\xVec)\) \ref{eq:Min_smoothFunc} could restrain too much the closed-loop system in comparison to theoretical \glsxtrshort{CBF} \( \lim_{k \to \infty} h(\xVec)\). In order to relax the safe set \(\mathcal{C}\) and potentially contain \( \bigcap_{i \in \mathcal{I}} \mathcal{C}_i\): 
    
    \begin{equation}
        h(\xVec) = \frac{-1}{k}ln(\sum_{i \in \mathcal{I}} e^{-k \hspace{0.2em} h_i(\xVec)}) + \frac{b}{k}
        \label{eq:Min_smoothFunc_tighted} \\  
    \end{equation}



    Beeing \(b\) a parameter tuned according with the problem needs, knowing that if \(b = ln (N)\):

    \begin{subequations}
        \begin{align}
            &\underset{i \in \mathcal{I}}{\min} (h_i(\xVec)) \leq h(\xVec) % \leq \underset{i \in\mathcal{I}}{\max} + \frac{b}{k} \quad \forall \xVec \in \mathbb{R}^n
            \label{eq:Intersection_CBF_Comparation_tighted} \\
            &\mathcal{C} \supseteq \bigcap_{i \in \mathcal{I}} \mathcal{C}_i
            \label{eq:Intersection_SafeSet_Comparation_tighted}
        \end{align}
    \end{subequations}

    Making the closed-loop system less constrained but not as safe.

    \item[Union of Sets \(\setminus\) OR \(\setminus\) Maximum]

    \begin{align}
        h(\xVec) &= \frac{1}{k}ln(\sum_{i \in \mathcal{I}} e^{k \hspace{0.2em} h_i(\xVec)}) 
        \label{eq:Max_smoothFunc} \\
                 &\approx \underset{i \in \mathcal{I}}{\max} (h_i(\xVec))
        \notag  
    \end{align}

    Beeing \(k \in \mathbb{R}_{>0}\) as smoothing parameter. The derivative of \(h(\xVec)\) is equal to \ref{eq:Min_smooth_derivative} but:

    \begin{equation}
        \lambda_i(\xVec) = e^{k(h_i(\xVec)-h(\xVec))}
        \label{eq:Max_lambda_smoothFunc}
    \end{equation}

    With also \(\sum_{i \in \mathcal{I}} \lambda_i(\xVec) = 1\). \par
    Due to an over-aproximation of \(h\) relative to the \(\max\) function, given the safe set \(\mathcal{C}\) defined based on \ref{eq:safe-set} it verifies:

    \begin{subequations}
        \begin{align}
            &\underset{i \in \mathcal{I}}{\max} (h_i(\xVec)) \leq h(\xVec) \quad \forall \xVec \in \mathbb{R}^n 
            \label{eq:Union_CBF_Comparation} \\
            &\mathcal{C} \supseteq \bigcup_{i \in \mathcal{I}} \mathcal{C}_i
            \label{eq:Union_SafeSet_Comparation}
        \end{align}
    \end{subequations}

    But, \(\hspace{0.4em} \lim_{k \to \infty} h(\xVec) = {\max}_{i \in \mathcal{I}} (h_i(\xVec)) \hspace{0.4em}\) and \( \hspace{0.4em} \lim_{k \to \infty} \mathcal{C} = \bigcup_{i \in \mathcal{I}} \mathcal{C}_i\) \par

    Now showing a different point of view, safety is a critical factor, so in order to tight the safe set \(\mathcal{C}\) and potentially be contained within \( \bigcup_{i \in \mathcal{I}} \mathcal{C}_i\): 
    
    \begin{equation}
        h(\xVec) = \frac{1}{k}ln(\sum_{i \in \mathcal{I}} e^{k \hspace{0.2em} h_i(\xVec)}) - \frac{b}{k}
        \label{eq:Max_smoothFunc_tighted} \\  
    \end{equation}

    Beeing \(b\) a tunable parameter. Taking \(b = ln (N)\):

    \begin{subequations}
        \begin{align}
            &\underset{i \in \mathcal{I}}{\max} (h_i(\xVec)) \geq h(\xVec) % \leq \underset{i \in\mathcal{I}}{\max} + \frac{b}{k} \quad \forall \xVec \in \mathbb{R}^n
            \label{eq:Union_CBF_Comparation_tighted} \\
            &\mathcal{C} \subseteq \bigcup_{i \in \mathcal{I}} \mathcal{C}_i
            \label{eq:Union_SafeSet_Comparation_tighted}
        \end{align}
    \end{subequations}

    Therefore the higher \(b\), the safer the system~\ref{eq:NL-CL-System-0Backstep} but possibly too constrained by the new \glsxtrshort{CBF} \(h(\xVec)\) \ref{eq:Max_smoothFunc_tighted}. 
\end{description}



\subsubsection{Second-Order System}
\label{subsub:uniCBF_secondOrder_system}

Following the explanations given in the previous subsubsection~\ref{subsub:uniCBF_firstOrder_system}, now applying for a second-order system~\ref{eq:HH-NL-System}, more precisely, for a Backstepping~\ref{sub:backstepping} situation. \\

Given a \glsxtrshort{CLF} \(V_0(\xVec): \) and multiple \glsxtrshort{CBF} \(h0_i(\xVec) \quad \forall i \in \mathcal{I}\) combined into a \glsxtrshort{CBF} \(h_0(\xVec)\) via \ref{eq:Min_smoothFunc_tighted} or \ref{eq:Max_smoothFunc_tighted} (according it is an AND or OR operation), through the \glsxtrshort{QP}~\ref{eq:CLF-CBF-QP-relaxed-0Backstep} (or its closed-form solution~\ref{eq:closed-form_controller}) is obtained a differentiable and locally Lipschitz continuos theoretical controller \(k_0(\xVec): \mathbb{R}^n \to \mathbb{R}\). \\

In order to the second-order system~\ref{eq:HH-NL-CL-System} dynamic converge to \(k_0(\xVec)\) imposed dynamic, given multiple \glsxtrshort{CBF} \(h_i(\xVec) \quad \forall i \in \mathcal{I} \):

\begin{subequations}
    \label{eq:Backstepp_CF_multipleCBF}
    \begin{align}
        &V_{\mathbf{\xi}}(\xVec, \mathbf{\xi}) = \frac{1}{2}|| \mathbf{\xi} -  k_0(\xVec) ||^2 
        \label{eq:Backstepp_prejudice_multipleCBF}\\
        &V(\xVec, \mathbf{\xi}) = V_0(\xVec) + V_{\mathbf{\xi}}(\xVec, \mathbf{\xi}) 
        \label{eq:Backstepp_V_multipleCBF}\\
        &h_i(\xVec, \mathbf{\xi}) = h0_i(\xVec) - V_{\mathbf{\xi}}(\xVec, \mathbf{\xi}) \quad \forall i \in \mathcal{I}
        \label{eq:Backstepp_h_multipleCBF}
    \end{align}
\end{subequations}

The resultant \glsxtrshort{CBF} \(h_i(\xVec, \mathbf{\xi}) \quad \forall i \in \mathcal{I}\), one more time can be merged via \ref{eq:Min_smoothFunc_tighted} or \ref{eq:Max_smoothFunc_tighted} (according it is an AND or OR operation) into a single \glsxtrshort{CBF} \(h(\xVec, \mathbf{\xi})\). However there is another way in this case to have the new single \glsxtrshort{CBF} \(h(\xVec, \mathbf{\xi})\), so, considering the second-order system~\ref{eq:HH-NL-System} and \(lo \in \{\max, \min\}\):

\begin{align}
        \underset{ i \in \mathcal{I}}{lo}(h0_i(\xVec) - V_{\mathbf{\xi}}(\xVec, \mathbf{\xi})) &= \underset{ i \in \mathcal{I}}{lo}(h0_i(\xVec)) - V_{\mathbf{\xi}}(\xVec, \mathbf{\xi}) \approx 
                                                                                                \notag\\
                                                                                               &\approx h_0(\xVec) - V_{\mathbf{\xi}}(\xVec, \mathbf{\xi}) = 
                                                                                               \notag\\
                                                                                               &=h(\xVec, \mathbf{\xi})
                                                                                               \label{eq:Backstepp_h_multipleCBF2}
\end{align}

\begin{equation}
        \dot{h}(\xVec, \mathbf{\xi}) = \underbrace{L_fh_0(\xVec) + L_Gh_0(\xVec)\mathbf{\xi} - L_fV_{\mathbf{\xi}}(\xVec, \mathbf{\xi})}_{L_fh(\xVec, \mathbf{\xi})} + (\underbrace{- L_GV_{\mathbf{\xi}}(\xVec, \mathbf{\xi})}_{L_Gh(\xVec, \mathbf{\xi})}) \uVec
        \label{eq:Backstepp_h_multipleCBF_derivatives2}
\end{equation}


Having \glsxtrshort{CBF} \(h\), the second-order solution, a differentiable and locally Lipschitz continuos controller \(k(\xVec, \mathbf{\xi}): \mathbb{R}^n \times \mathbb{R}^p \to \mathbb{R}\) is obtained using the \glsxtrshort{QP}~\ref{eq:CLF-CBF-QP-relaxed-1Backstep} (or its closed-form solution~\ref{eq:closed-form_controller}). \\

\begin{tcolorbox}[colback=blue!5!white,colframe=blue!35!white,title=Notes:]
\begin{itemize}
    \item There are other possible variations of a second-order system that invalidates \ref{eq:Backstepp_h_multipleCBF_derivatives2} like the unicycle model that will be shown next chapter.
\end{itemize}
\end{tcolorbox} 



\newpage %Só para ajeitar





















































\endinput































% \begin{tcolorbox}[colback=red!5!white,colframe=red!75!black,title=My Heading]
% This is a \textbf{tcolorbox}.
% \tcblower
% Here, you see the lower part of the box.
% \end{tcolorbox}



% \begin{center}
%   \fbox{\LARGE
%     This manual is outdated and must be revised!}
% \end{center}



% \begin{flushleft}
% \hspace*{0.5cm}“\verb!n015002t.ttf!”, “\verb!n015003t.ttf!”, and “\verb!n015006t.ttf!”
% \end{flushleft}
   

% \ref{it:project_available} above in Section~\ref{sub:with_a_local_latex_installation} (\nameref{sub:with_a_local_latex_installation}).


% subsection with_a_remote_cloud_based_service (end)


% \newcommand{\accessAllowed}{\includegraphics[align=c,width=1.9em]{access_allowed}}
% \newcommand{\accessForbiden}{\includegraphics[align=c,width=1.9em]{dont_touch}}
% \newcommand{\File}{\includegraphics[align=c,width=1.9em]{file}}
% \newcommand{\Folder}{\includegraphics[align=c,width=1.9em]{folder}}


% \bgroup
%     \rowcolors{1}{}{GhostWhite}
%       \begin{xltabular}{\textwidth}{>{\ttfamily}l>{\itshape}lcX}
%         \caption{The folders and files (top level).}
%         \label{tab:folders_and_files}\\
%         \toprule
%         \rowcolor{Gainsboro}%
%         Name & Type & Access & Contents \\
%         \midrule
% template.tex      & \File    & \accessForbiden &
% The main template file. You need to \emph{compile} this file with one of \pdfLaTeX, \XeLaTeX, or \LuaLaTeX\ to obtain the PDF file (”\texttt{template.pdf}”).  I recommend the usage of the ”\texttt{latexmk}” command or, if you use a UN*X-like OS, you may use ”\texttt{make}” (and the ggiven ”\texttt{Makefile}”).
% \\
% Config          & \Folder  & \accessAllowed &
% Configuration files.  Please customize your template by changing the files in this folder!
% \\
%         \bottomrule
%         \end{xltabular}
%     % \end{longtblr}
% \egroup


% \newcommand{\classoption}[4]{\textbf{#1=OPT}\newline\emph{\small#2}&\textbf{#3}\newline{\small#4}\\}
% \newcommand{\defaultopt}[1]{\mbox{$\Rightarrow$~\emph{Default: \texttt{#1}}}\newline}
% \newcommand{\defaultit}[1][default]{($\Leftarrow$~\emph{#1})}


% \bgroup
% \begin{xltabular}{\linewidth}{>{\hsize=.4\hsize\raggedright\arraybackslash}X>{\hsize=.6\hsize}X}
%   \toprule
% %----------------------------------------------------------------------
%   \classoption{doctype}%
%     {phd, phdprop, phdplan, msc, mscplan, bsc, plain}%
%     {The type of the document.}%
% 	{%
%     \begin{tabular}{@{}r@{ $\rightarrow$ }l@{}}
%         phd & PhD thesis \defaultit.\\
%     phdprop & PhD thesis proposal (for FCT-NOVA).\\
%     phdplan & PhD thesis plan.\\
%         msc & MSc thesis.\\
%     mscplan & MSc thesis plan.\\
%         bsc & BSc report.\\
%       plain & Other report.\\
%     \end{tabular}
%     }
% %----------------------------------------------------------------------
%     \midrule
%   \classoption{school}%
%   	{nova/fct
% 	}%
%     {Selection of the university and of the school (and degree variant).}%
%     {\defaultopt{school=nova/fct} }
% %----------------------------------------------------------------------
%     \midrule
%   \classoption{docstatus}%
%     {draft, provisional, final}%
%     {The current status of the document.}%
% 	{}
% %----------------------------------------------------------------------
%     \bottomrule
% \end{xltabular}
% \egroup


% \printbibliography[heading=subbibliography, segment=\therefsegment, title={\bibname\ for chapter~\thechapter}]

% \section{\glsfmtshort{novathesisclass}\ Class Options}
% \label{sec:package_options}

 % \todo[inline]{A a note in a line by itself.}







