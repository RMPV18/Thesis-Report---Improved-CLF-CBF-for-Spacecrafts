%!TEX root = ../template.tex
%%%%%%%%%%%%%%%%%%%%%%%%%%%%%%%%%%%%%%%%%%%%%%%%%%%%%%%%%%%%%%%%%%%%
%% chapter2.tex
%% NOVA thesis document file
%%
%% Chapter with the template manual
%%%%%%%%%%%%%%%%%%%%%%%%%%%%%%%%%%%%%%%%%%%%%%%%%%%%%%%%%%%%%%%%%%%%

\typeout{NT FILE chapter2.tex}%

\chapter{Methodology}
\label{cha:methodology}

\glsresetall %Para em vez de mostrar o acrónimo voltar a mostrar o nome completo para relembrar



%Aqui falar do modelo generico e de algumas coisas que serviram de base para os proximos topicos e que não valem a pena uma secção, alem de introduzir o que ai vem

\section{\glsxtrshort{CLF-CBF}}
\label{sec:clf_cbf}


\subsection{1st Order Formulation}
\label{sub:formulation}

\subsubsection{\glsxtrfull{CLF}}
\label{subsub:control_lyapunov_function}

Having the objective of global assymptotically stability of the system~\ref{eq:}, is desired for the system converge to an equilibrium point \( \bar{\mathbf{x}} \). A solution comes by designing a positive definite function \( V: \mathbb{R}^n \rightarrow \mathbb{R}_{\geq 0}  \) which converges to zero while the system converges to the respective equilibrium point, and for that, designing a control law that acts in order to achieve specifically the convergence of the function. Consequentially is pretended to the function \( V\mathbf(x) \) to be continuos and differentiable:

\[\dot{V} (\mathbf{x}, \mathbf{u}) = \frac{\partial V}{\partial x}\dot{x} = L_fV(\mathbf{x}) + L_GV(\mathbf{x})\mathbf{u} \]

where \(L_fV\) and \(L_GV\) are the Lie derivatives of \(V\) and \(\mathbf{u}\) the control input vector resposible to manipulate the function, more precisely, its convergence rate.\par

That beeing said, a continuosly differentiable, positive definite function \( V: \mathbb{R}^n \rightarrow \mathbb{R}_{\geq 0}  \) wich \( \lim_{\mathbf{x} \to \bar{\mathbf{x}}}{V(\mathbf{x})} = 0 \) is a \glsxtrshort{CLF} if there exists a \( \gamma \in \mathcal{K}^e_{\infty}  \)\footnote{METER AQUI UMA MENSAGEM A RODAPE A EXPLICAR ISTO}  such that for all \( \mathbf{x} \in \mathbb{R}^n \setminus \{\bar{\mathbf{x}}\} \):

\begin{equation}
 \inf_{\mathbf{u} \in \mathbb{R}^m} [L_fV(\mathbf{x}) + L_GV(\mathbf{x})\mathbf{u}] \leq -\gamma(V(\mathbf{x}))
 \label{eq:CLF}
\end{equation}

From \glsxtrshort{CLF} \(V\) and an adequate tuning of \( \gamma \) parameter results a control law followed by the controller \(K_{CLF}\):

\begin{equation}
 K_{CLF}(\mathbf{x}) = \{ \mathbf{u} \in \mathbb{R}^m: L_fV(\mathbf{x}) + L_GV(\mathbf{x})\mathbf{u} \leq -\gamma(V(\mathbf{x})) \}
 \label{eq:K-CLF}
\end{equation}

The stets generated by \(K_{CLF}(\mathbf{x}) \hspace{0.5em} \forall x \in \mathbb{R}^n \) refer to the possible control input values necessary to the \glsxtrshort{CLF} \(V(\mathbf{x})\) continuosly converge to zero while the closed-loop system~\ref{} to the equilibrium point, but more slowly as is close to reach the objective.


\subsubsection{\glsxtrfull{CBF}}
\label{subsub:control_barrier_function}

Now having the objective of garantee safety to system~\ref{}, is aimed to keep the state inside a safe set \(\mathcal{C}\) (foward invariant\footnote{METER AQUI EXPLICAÇÂO}) defined accordingly the problem needs. The notion of safety, i.e., the safe set \(\mathcal{C}\) is defined by the condition of a differentiable function \( h: \mathbb{R}^n \rightarrow \mathbb{R}  \):

\begin{equation}
    \begin{array}{l}
        \qquad   \mathcal{C} = \{ \mathbf{x} \in \mathbb{R}^n : h( \mathbf{x} )\geq 0\} \\
        \qquad \hspace{-0.5em}  \partial\mathcal{C} = \{ \mathbf{x} \in \mathbb{R}^n : h( \mathbf{x} ) = 0\} \\
        int(\mathcal{C}) = \{ \mathbf{x} \in \mathbb{R}^n : h( \mathbf{x} ) > 0\}
    \end{array}
 \label{eq:safe-set}
\end{equation}


In order to mantain \(h( \mathbf{x} )\) positive, there is a need to have a control law and but first to be able to interface with the function which is the case since \(h( \mathbf{x} )\) is differentiable:  

\[\dot{h}(\mathbf{x}, \mathbf{u}) = \frac{\partial h}{\partial x}\dot{x} = L_fh(\mathbf{x}) + L_Gh(\mathbf{x})\mathbf{u} \]

where \(L_fh\) and \(L_Gh\) are the Lie derivatives of \(h\) and \(\mathbf{u}\) the control input vector resposible to manipulate the function, more precisely, its time derivative.\par

So, a differentiable function \( h: \mathbb{R}^n \rightarrow \mathbb{R}  \) is a \glsxtrshort{CBF} if there exists a \( \alpha \in \mathcal{K}^e_{\infty}  \) such that for all \( \mathbf{x} \in \mathbb{R}^n \setminus \{\bar{\mathbf{x}}\} \):

\begin{equation}
 \sup_{\mathbf{u} \in \mathbb{R}^m} [L_fh(\mathbf{x}) + L_Gh(\mathbf{x})\mathbf{u}] \geq -\alpha(h( \mathbf{x} ))
 \label{eq:CBF}
\end{equation}

Similar to the \glsxtrshort{CLF}, from \glsxtrshort{CBF} \(h\) and an adequate tuning of \( \alpha \) parameter results a control law encapsulated by the controller \(K_{CBF}\):

\begin{equation}
 K_{CBF}(\mathbf{x}) = \{ \mathbf{u} \in \mathbb{R}^m: L_fh(\mathbf{x}) + L_Gh(\mathbf{x})\mathbf{u} \geq -\alpha(h( \mathbf{x} )) \}
 \label{eq:K-CBF}
\end{equation}


The controller \(K_{CLF}(\mathbf{x}) \hspace{0.5em} \forall x \in \mathbb{R}^n \) renders control input sets that keep the closed-loop system safe ( relative to the formulation of the function \(h\) ). The controller by defenition is able to evolve the closed-loop system~\ref{} more freely when \(\dot{h}( \mathbf{x} ) > 0\) and more cautly as \(h( \mathbf{x} ) \) shortens and \(\dot{h}( \mathbf{x} ) < 0\), if \(h( \mathbf{x} ) < 0\) it means is in the unsafe region so \(\dot{h}( \mathbf{x}) > 0\) converging continuosly to the safe set \(\mathcal{C}\). 




\subsubsection{Quadratic Program Formulation}
\label{subsub:quadratic_program_formulation}

Now having the objective of not only achieve safety but also assymptotically stability through the usage of \glsxtrshort{CLF} and \glsxtrshort{CBF} it renders a set of control inputs able to achieve the respective purposes.

Using the conditions imposed by \glsxtrshort{CLF} and \glsxtrshort{CBF} as constraints, is presented an optimization problem  via \glsxtrfull{QP}:

%meter formula


%Meter aqui no final uma nota com uma caixinha azul a dizer que se  pode arbitrariamente não usar as CLF ou CBF se nao se pretende uobter establidade ou segurança e que a formulação continua sendo igual fora a remoção da condição
%, and hopefully non-empty which can happen if in conflict with other conditions.
%Slack variable
%tikz daquele esquema


\subsection{Backstepping}
\label{sub:backstepping}

\subsection{Closed Form}
\label{sub:closed_form}























\endinput




% \begin{center}
%   \fbox{\LARGE
%     This manual is outdated and must be revised!}
% \end{center}



% \begin{flushleft}
% \hspace*{0.5cm}“\verb!n015002t.ttf!”, “\verb!n015003t.ttf!”, and “\verb!n015006t.ttf!”
% \end{flushleft}
   

% \ref{it:project_available} above in Section~\ref{sub:with_a_local_latex_installation} (\nameref{sub:with_a_local_latex_installation}).


% subsection with_a_remote_cloud_based_service (end)


% \newcommand{\accessAllowed}{\includegraphics[align=c,width=1.9em]{access_allowed}}
% \newcommand{\accessForbiden}{\includegraphics[align=c,width=1.9em]{dont_touch}}
% \newcommand{\File}{\includegraphics[align=c,width=1.9em]{file}}
% \newcommand{\Folder}{\includegraphics[align=c,width=1.9em]{folder}}


% \bgroup
%     \rowcolors{1}{}{GhostWhite}
%       \begin{xltabular}{\textwidth}{>{\ttfamily}l>{\itshape}lcX}
%         \caption{The folders and files (top level).}
%         \label{tab:folders_and_files}\\
%         \toprule
%         \rowcolor{Gainsboro}%
%         Name & Type & Access & Contents \\
%         \midrule
% template.tex      & \File    & \accessForbiden &
% The main template file. You need to \emph{compile} this file with one of \pdfLaTeX, \XeLaTeX, or \LuaLaTeX\ to obtain the PDF file (”\texttt{template.pdf}”).  I recommend the usage of the ”\texttt{latexmk}” command or, if you use a UN*X-like OS, you may use ”\texttt{make}” (and the ggiven ”\texttt{Makefile}”).
% \\
% Config          & \Folder  & \accessAllowed &
% Configuration files.  Please customize your template by changing the files in this folder!
% \\
%         \bottomrule
%         \end{xltabular}
%     % \end{longtblr}
% \egroup


% \newcommand{\classoption}[4]{\textbf{#1=OPT}\newline\emph{\small#2}&\textbf{#3}\newline{\small#4}\\}
% \newcommand{\defaultopt}[1]{\mbox{$\Rightarrow$~\emph{Default: \texttt{#1}}}\newline}
% \newcommand{\defaultit}[1][default]{($\Leftarrow$~\emph{#1})}


% \bgroup
% \begin{xltabular}{\linewidth}{>{\hsize=.4\hsize\raggedright\arraybackslash}X>{\hsize=.6\hsize}X}
%   \toprule
% %----------------------------------------------------------------------
%   \classoption{doctype}%
%     {phd, phdprop, phdplan, msc, mscplan, bsc, plain}%
%     {The type of the document.}%
% 	{%
%     \begin{tabular}{@{}r@{ $\rightarrow$ }l@{}}
%         phd & PhD thesis \defaultit.\\
%     phdprop & PhD thesis proposal (for FCT-NOVA).\\
%     phdplan & PhD thesis plan.\\
%         msc & MSc thesis.\\
%     mscplan & MSc thesis plan.\\
%         bsc & BSc report.\\
%       plain & Other report.\\
%     \end{tabular}
%     }
% %----------------------------------------------------------------------
%     \midrule
%   \classoption{school}%
%   	{nova/fct
% 	}%
%     {Selection of the university and of the school (and degree variant).}%
%     {\defaultopt{school=nova/fct} }
% %----------------------------------------------------------------------
%     \midrule
%   \classoption{docstatus}%
%     {draft, provisional, final}%
%     {The current status of the document.}%
% 	{}
% %----------------------------------------------------------------------
%     \bottomrule
% \end{xltabular}
% \egroup


% \printbibliography[heading=subbibliography, segment=\therefsegment, title={\bibname\ for chapter~\thechapter}]

% \section{\glsfmtshort{novathesisclass}\ Class Options}
% \label{sec:package_options}

 % \todo[inline]{A a note in a line by itself.}







