%!TEX root = ../template.tex
%%%%%%%%%%%%%%%%%%%%%%%%%%%%%%%%%%%%%%%%%%%%%%%%%%%%%%%%%%%%%%%%%%%%
%% abstract-en.tex
%% NOVA thesis document file
%%
%% Abstract in English
%%%%%%%%%%%%%%%%%%%%%%%%%%%%%%%%%%%%%%%%%%%%%%%%%%%%%%%%%%%%%%%%%%%%

\typeout{NT FILE abstract-en.tex}%


This thesis presents a set of low-effort \glsxtrshort{CLF-CBF} based frameworks that via optimization of a single parameter of the \glsxtrshort{CBF} condition is capable to have a better fit to the variable encountered scenarios. Among the presented control techniques, it is also shown how changing the stability objective to a transitive computed point on the boundary of the safe set prevents some potential convergence to unwanted equilibrium points and massively improves the path taken, as it mitigates \glsxtrshort{CLF-CBF} unnecessary approximations to the unsafe set that lead to larger control inputs, without diminishing its speed of convergence to the desired equilibrium point. The controllers are tested in different collision scenarios and for different systems, including a unicycle and spacecraft. 







%%% Comandos que Podem Dar Jeito

% \verb+5_packages.tex+.


% \begin{verbatim}
%     \ntsetup{abstractorder={de={de,en,it}}}
% \end{verbatim}


% \begin{enumerate}
%   \item What is the problem?
%   \item Why is this problem interesting/challenging?
%   \item What is the proposed approach/solution/contribution?
%   \item What results (implications/consequences) from the solution?
% \end{enumerate}

% Palavras-chave do resumo em Inglês
% \begin{keywords}
% Keyword 1, Keyword 2, Keyword 3, Keyword 4, Keyword 5, Keyword 6, Keyword 7, Keyword 8, Keyword 9
% \end{keywords}
