%!TEX root = ../template.tex
%%%%%%%%%%%%%%%%%%%%%%%%%%%%%%%%%%%%%%%%%%%%%%%%%%%%%%%%%%%%%%%%%%%%
%% abstract-en.tex
%% NOVA thesis document file
%%
%% Abstract in English
%%%%%%%%%%%%%%%%%%%%%%%%%%%%%%%%%%%%%%%%%%%%%%%%%%%%%%%%%%%%%%%%%%%%

\typeout{NT FILE abstract-en.tex}%



% This thesis focus on the development of low effort control strategies based on \glsxtrfull{CLF-CBF} for collision avoidance scenarios. The \glsxtrshort{CLF-CBF} is a fast computing method, that provides safety and directs the system towards a desired location, however, this control method can get the system relatively close to the obstacles, possibly leading to undesired equilibriums near them. More than that, it reveals a lack of adaptability, reflecting in poor trajectories avoiding different obstacles. \par   
% This dissertation presents a set of low-effort \glsxtrshort{CLF-CBF} based frameworks that via optimization of a single parameter of the \glsxtrshort{CBF} condition is capable to have a better fit to the variable encountered scenarios. Among the presented control techniques, it is also shown how changing the stability objective to a transitive computed point on the boundary of the obstacle prevents some potential convergence to unwanted equilibrium points and massively improves the path taken, as it mitigates \glsxtrshort{CLF-CBF} unnecessary approximations to the unsafe set that lead to larger control inputs, without diminishing its speed of convergence to the desired equilibrium point.\par
% The controllers are tested in different collision scenarios and for different systems, including a unicycle and spacecraft. 

The focal point of this thesis is the development of low-effort control strategies based on \glsxtrfull{CLF-CBF} for collision avoidance scenarios. The \glsxtrshort{CLF-CBF} has been demonstrated to be a rapid computing method that provides safety and directs the system towards a desired location. However, this control method has been observed to bring the system relatively close to obstacles, which may result in undesired equilibria in their vicinity. Furthermore, this approach has been shown to reveal a lack of adaptability, as evidenced by the poor trajectories that are adopted in order to avoid varied obstacles.\par
The present dissertation sets forth a series of low-effort \glsxtrshort{CLF-CBF} frameworks. These frameworks are capable of achieving a superior fit to variable encountered scenarios by optimising a single parameter of the \glsxtrshort{CBF} condition. In the context of the presented control techniques, it is demonstrated that modifying the stability objective to a transitive computed point on the boundary of the obstacle can prevent convergence to undesirable equilibrium points and significantly enhance the trajectory. This is achieved by mitigating \glsxtrshort{CLF-CBF} unnecessary approximations to the obstacle, which result in larger control inputs, without compromising the rate of convergence to the desired equilibrium point. \par
The controllers are subjected to rigorous testing in a range of collision scenarios and across diverse systems, encompassing unicycles and spacecraft.





%%% Comandos que Podem Dar Jeito

% \verb+5_packages.tex+.


% \begin{verbatim}
%     \ntsetup{abstractorder={de={de,en,it}}}
% \end{verbatim}


% \begin{enumerate}
%   \item What is the problem?
%   \item Why is this problem interesting/challenging?
%   \item What is the proposed approach/solution/contribution?
%   \item What results (implications/consequences) from the solution?
% \end{enumerate}

% Palavras-chave do resumo em Inglês
% \begin{keywords}
% Keyword 1, Keyword 2, Keyword 3, Keyword 4, Keyword 5, Keyword 6, Keyword 7, Keyword 8, Keyword 9
% \end{keywords}
