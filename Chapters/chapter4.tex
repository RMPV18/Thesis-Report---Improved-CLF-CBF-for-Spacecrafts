%!TEX root = ../template.tex
%%%%%%%%%%%%%%%%%%%%%%%%%%%%%%%%%%%%%%%%%%%%%%%%%%%%%%%%%%%%%%%%%%%%
%% chapter4.tex
%% NOVA thesis document file
%%
%% Chapter with lots of dummy text
%%%%%%%%%%%%%%%%%%%%%%%%%%%%%%%%%%%%%%%%%%%%%%%%%%%%%%%%%%%%%%%%%%%%

\typeout{NT FILE chapter4.tex}%

\chapter{Conclusion}
\label{cha:conclusion}

Safety is a dominant topic in control, among its areas of application there is usually a set of states which is crucial the system not to evolve. In the context of autonomous vehicles collision avoidance is a critical factor, as safety is prioritized over performance. But a poor performance could mean unfeasibility of the respective system for a real-life scenarios, growing the need to safe techniques increasingly better, which the purposed algorithms in this thesis pretend to answer. In the space industry safety and reliability are predominant, since missions are costly and subject to long deadlines, that together with the inaccessible locations and desire to move further away from earth lead to the automation of spacecrafts, controlled by safe techniques but welcome to better performances and to low-cost techniques computationally wise aiming costs reductions and deal with the limitations of the spacecrafts processors.


Given a low-cost \glsxtrshort{CLF-CBF} framework, this thesis presented essentially two techniques able to keep the system safe (\glsxtrshort{A-JO} and \glsxtrshort{A-CLF-S}). Both of them adaptable to different scenarios via optimization of the \(\alpha\) parameter of the \glsxtrshort{CBF} to improved performance. The optimization of the repective variables was done applying a binary search algorithm, preceded by a gradient descent method aiming to found an interval where a local minimum of the parameter resided. The gradient descent method is charactherized by an constant responsible to adjust the update step, chosen to be variable, normalizing the resulting step as the cost specter was noisy and with high amplitudes, leading easily to impractical \(\alpha\). 


The \glsxtrshort{A-CLF-S} differentiates itself by the change of stability objective to a new computed waypoint on the boundary of the safe, and like this preventing the characteristic \glsxtrshort{CLF-CBF} undesired approximation to the unsafe sets and consequent high control inputs aiming to keep the system safe while on track with the soften objective of convergence to the equilibrium point. However, the sudden change of objective of the \glsxtrshort{A-CLF-S} is propitious to more abrupt maneuvers and therefore high control inputs.   

It was verified that the \glsxtrshort{A-JO}, thanks to its based framework, given certain conditions converges to the boundary, which is fought by the \glsxtrshort{A-CLF-S} converging to the obtained point in the boundary. Like \glsxtrshort{A-JO}, in the presented simulation on this thesis, \glsxtrshort{MPC} also converged to the boundary, mainly owing to its limited prediction horizon but also because of its objective function aiming to approximate all system sates along the horizon to the reference instead of actually optimizing the reaching time. By changing the sate cost in objective function just to the last sate from the horizon it is possible to mitigate the problem, however in case of a too big horizon reaching the reference is delayed to the end of it as the controller targets the last instant of the horizon, but with even lower input cost. Giving space to the formulation of increasingly weighted sates along the horizon. Following that strategy, the cost function used by \glsxtrshort{A-CLF-S} during optimization is calculated purely based on the last state from its horizon too, but due to its baseline framework, in particular to the static \(\gamma\) the paths generated don't suffer from the same event of the \glsxtrshort{MPC}, therefore the respective cost functions helps achieving a better path.       

That beeing said, giving the low-computation effort, along with the safety and performance objective, the \glsxtrshort{A-CLF-S} showed consistently better results than \glsxtrshort{A-JO} although the relative higher computational times. Compared to \glsxtrshort{MPC}, it presented the more positive results, relatively close performance wise but particularly lower computationally, around two and three orders of magnitude. The purposed techniques validated for different systems and scenarios, a unicycle against circular and ellipsoidal obstacles, and a spacecraft in a keep in formation scenario demanding contour other spacecraft delimited by a polytope.  

The \glsxtrshort{CLF-CBF} based algorithms suffer from slower convergence as it approaches the desired state so future work passes by correcting it for example implementing a \glsxtrshort{LQR} when there is no eminent possibility of collision with the unsafe set. 

Another way to continue this project is by devoloping an an waypoint obtaining algorithm to a three dimensional unsafe set and make it viable to more applications like in space. 

In order to keep the insertion and validity of these techniques in the space industry, implementing them as high frequency controllers, trying to keep on track the spacecrafts with well known maneuvers, while keeping the system safe against debris with minimal intrusion.  

In addition, applied them initially more realistic scenarios, by simulating them along with a lidar technology for object detection and even with moving obstacles.

Finally, for future work, could potentially pass by use them in real life systems. 