%!TEX root = ../template.tex
%%%%%%%%%%%%%%%%%%%%%%%%%%%%%%%%%%%%%%%%%%%%%%%%%%%%%%%%%%%%%%%%%%%%
%% chapter4.tex
%% NOVA thesis document file
%%
%% Chapter with lots of dummy text
%%%%%%%%%%%%%%%%%%%%%%%%%%%%%%%%%%%%%%%%%%%%%%%%%%%%%%%%%%%%%%%%%%%%

\typeout{NT FILE chapter4.tex}%

\chapter{Conclusion}
\label{cha:conclusion}

\section{Summary}
\label{sec:summary}

%Safety is a dominant topic in control, among its areas of application there is usually a set of states which is crucial for the system not to evolve. In the context of autonomous vehicles collision avoidance is a critical factor, as safety is prioritized over performance. But a poor performance could mean unfeasibility of the respective system for real-life scenarios, growing the need to safe techniques increasingly better, which the proposed algorithms in this thesis pretend to answer. In the space industry safety and reliability are predominant, since missions are costly and subject to long deadlines, that together with the inaccessible locations and desire to move further away from earth lead to the automation of spacecrafts, controlled by safe techniques but welcome to better performances and to low-cost techniques computationally wise aiming costs reductions and deal with the limitations of the spacecrafts processors.
Safety is a prevailing theme within the domain of control. Among the various domains of application of control, there is typically a set of states that is of importance in ensuring that the system does not evolve. In the context of autonomous vehicles, collision avoidance is a critical factor, with safety being prioritised over performance. Nevertheless, inadequate performance could result in the respective system being deemed impractical for real-life scenarios, thereby creating a growing need for increasingly sophisticated safety techniques. The proposed algorithms in this thesis are intended to address this need. In the domain of space industry, issues of safety and reliability assume paramount importance. The inherent high costs of space missions, coupled with the stringent deadlines and the remote nature of space, necessitate the automation of spacecraft operations. These spacecraft controlled by safe techniques that are conducive to enhanced performance, while aiming computational cost-effectiveness and addressing the limitations of spacecraft processors.

%Given a low-cost \glsxtrshort{CLF-CBF} framework, this thesis presented essentially two techniques able to keep the system safe (\glsxtrshort{A-JO} and \glsxtrshort{A-CLF-S}). Both of them are adaptable to different scenarios via optimization of the \(\alpha\) parameter of the \glsxtrshort{CBF} to improve performance. The optimization of the respective variables was done usinga binary search algorithm, preceded by a gradient descent method aiming to findan interval where a local minimum of the parameter resided. The gradient descent method is characterized by a constant responsible to adjust the update step, chosen to be variable, normalizing the resulting step as the cost specter was noisy and with high amplitudes, leading easily to impractical \(\alpha\). 
Given a low-cost \glsxtrshort{CLF-CBF} framework, this thesis has presented two techniques that have the capacity to ensure the safety of the system (\glsxtrshort{A-JO} and \glsxtrshort{A-CLF-S}). It has been demonstrated that both of these systems are capable of adapting to different scenarios by optimising the alpha parameter of the CBF, thereby enhancing performance. The respective variables were optimised using a binary search algorithm, preceded by a gradient descent method that sought to identify an interval where a local minimum of the parameter was located. The gradient descent method is characterised by a parameter that is responsible for adjusting the update step. The selected parameter is intended to be variable, with the intention of normalising the resulting step. This is due to the fact that the cost spectrum was found to be noisy and characterised by high amplitudes, which could easily result in impractical values of alpha. 

%The \glsxtrshort{A-CLF-S} differentiates itself by the change of stability objective to a new computed waypoint on the boundary of the safe, and like this preventing the characteristic \glsxtrshort{CLF-CBF} undesired approximation to the unsafe sets and consequent high control inputs aiming to keep the system safe while on track with the soften objective of convergence to the equilibrium point. However, the sudden change of objective of the \glsxtrshort{A-CLF-S} is propitious to more abrupt maneuvers and therefore high control inputs.   
The \glsxtrshort{A-CLF-S} model is distinguished by its modification of the stability objective, whereby a novel computed waypoint is delineated on the boundary of the safe set. This modification serves to preclude the occurrence of the characteristic undesired approximation to the unsafe sets, which is characteristic of the  \glsxtrshort{CLF-CBF} framework. Consequently, high control inputs are required in order to ensure that the system remains on track with the softened objective of convergence to the equilibrium point, while maintaining safety. Nevertheless, the abrupt modification in the objective of the \glsxtrshort{A-CLF-S} is conducive to more abrupt manoeuvres and consequently high control inputs.   

%It was verified that the \glsxtrshort{A-JO}, thanks to its based framework, given certain conditions converges to the boundary, which is fought by the \glsxtrshort{A-CLF-S} converging to the obtained point in the boundary. Like \glsxtrshort{A-JO}, in the presented simulation on this thesis, \glsxtrshort{MPC} also converged to the boundary, mainly owing to its limited prediction horizon but also because of its objective function aiming to approximate all system states along the horizon to the reference instead of actually optimizing the reaching time. By changing the state cost in objective function just to the last state from the horizon it is possible to mitigate the problem, however in case of a too big horizon reaching the reference is delayed to the end of it as the controller targets the last instant of the horizon, but with even lower input cost. Giving space to the formulation of increasingly weighted states along the horizon. Following that strategy, the cost function used by \glsxtrshort{A-CLF-S} during optimization is calculated purely based on the last state from its horizon too, but due to its baseline framework, in particular to the static \(\gamma\) the paths generated don't suffer from the same event of the \glsxtrshort{MPC}, therefore the respective cost functions helps achieving a better path.       
It has been established that, under specific conditions, the \glsxtrshort{A-JO} (as defined by the framework) tends to approach the boundary, while the \glsxtrshort{A-CLF-S}  (as defined by the framework) approaches the obtained point on the boundary. As demonstrated in the simulation presented in this dissertation, the  \glsxtrshort{MPC} also converged to the boundary, primarily due to its limited prediction horizon, but also because of its objective function, which aimed to approximate all system states along the horizon to the reference, rather than actually optimizing the reaching time. It is possible to mitigate the problem by changing the state cost in the objective function to the last state from the horizon. However, in the event of an extensive horizon, the reference is delayed until the end of it, as the controller targets the final instant of the horizon, but with an even lower input cost. Consequently, the promotion of the formulation of increasingly state-weighted along the horizon is encouraged. In accordance with the aforementioned strategy, the cost function employed by \glsxtrshort{A-CLF-S}  during the optimisation process is calculated exclusively on the basis of the final state from its simulation horizon. However, due to its baseline framework, particularly the static gamma slope, the paths generated \glsxtrshort{A-CLF-S}  do not suffer from the same events as those of the \glsxtrshort{MPC} . Consequently, the respective cost functions facilitate the achievement of optimum paths.       

%That being said, giving the low-computation effort, along with the safety and performance objective, the \glsxtrshort{A-CLF-S} showed consistently better results than \glsxtrshort{A-JO} despite the relatively higher computational times. Compared to \glsxtrshort{MPC}, it presented the more positive results, relatively close performance wise but particularly lower computationally, around two and three orders of magnitude. The proposed techniques validated for different systems and scenarios, a unicycle against circular and ellipsoidal obstacles, and a spacecraft in a keep in formation scenario demanding contour other spacecraft delimited by a polytope.  
The proposed techniques were applied in indifferent systems and collision scenarios. The following scenarios are presented: a unicycle against circular and ellipsoidal obstacles, and a spacecraft in a keep-in formation scenario, where contouring other spacecraft is required, and these are delimited by a polytope.  Notwithstanding the comparatively reçatively higher computational times, the \glsxtrshort{A-CLF-S}  exhibited consistently superior results in comparison with the \glsxtrshort{A-JO} with regard to performance. In comparison with \glsxtrshort{MPC}, the results obtained were more positive. The performance of the former was relatively similar, but the latter was particularly more efficient in terms of computation, by a margin of two to three orders of magnitude. 

%The \glsxtrshort{CLF-CBF} based algorithms suffer from slower convergence as it approaches the desired state, so a solution comes by changing the controller. The \glsxtrshort{A-CLF-S} reveal in later stages some difficulty reaching the desired location, governed by a typical \glsxtrshort{CLF-CBF} which consistently show negative results compared to the \glsxtrshort{MPC} which has proven the ability to reach the equilibrium point.
As the \glsxtrshort{CLF-CBF}-based algorithms approach the desired state, they exhibit slower convergence. To address this issue, a solution is proposed that involves modifying the controller. The \glsxtrshort{A-CLF-S} reveal in later stages some difficulty attaining the desired location, governed by a typical \glsxtrshort{CLF-CBF} which consistently show negative results compared to the \glsxtrshort{MPC} which has proven the ability to reach the equilibrium point.


\section{Future Work}
\label{sec:future_work}

% Future work can pass by implementing a filter based on the cost evaluation as outliers, aiming to increase the accuracy and speed up the optimization algorithms.
% Another way to continue this project is by developing a waypoint obtaining algorithm to a three dimensional unsafe set and make it viable to more applications like in space. 
% In order to keep the insertion and validity of these techniques in the space industry, implementing them as high frequency controllers, trying to keep on track the spacecraft with well known maneuvers, while keeping the system safe against debris with minimal intrusion.  
% In addition, apply in more realistic scenarios, by simulating them along with a lidar technology for object detection and even with moving obstacles.
% Finally, for future work, could potentially pass by use them in real life systems. 

It is recommended that future research be conducted in order to implement a filter based on cost evaluation as outliers. The objective of this would be to increase the accuracy and accelerate the optimisation algorithms.
A potential approach to the continuation of this project would be the development of a waypoint-obtaining algorithm for a three-dimensional unsafe set, with the objective of enhancing its viability for a range of applications, including those in space. 
In order to maintain the incorporation and validity of these techniques within the domain of the space industry, a proposal is hereby submitted for their implementation as high-frequency low-level controllers. The objective of this proposal is, firstly to ensure the continued trajectory of spacecraft through the utilisation of well-known manoeuvres and secondly, to ensure the safety of the system against potential debris, while minimising the disruption to the system.  
Furthermore, the application should be extended to more realistic scenarios by simulating them with lidar technology for object detection and even with moving obstacles.
Ultimately, the utilisation of these systems in real-life applications is a potential avenue for future research endeavours. 